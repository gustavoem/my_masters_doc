\documentclass[tikz]{standalone}


\usepackage{tikz}
\usepackage{verbatim}
\usepackage{my_colors}
\usepackage{geometry}

\usetikzlibrary{shapes.misc} % rounded rectangle
\usetikzlibrary{shapes} % ellipse
\usetikzlibrary{calc} % ellipse

\begin{document}
%\pagestyle{empty}
%\def\layersep{2.5cm}

\begin{tikzpicture}[shorten >=1pt,->,draw=black!30, 
    node distance=\layersep]

    \useasboundingbox (0, -6) rectangle (10, 2);

        
    \footnotesize
    \tikzstyle{every pin edge}=[<-,shorten <=1pt]
    %\tikzstyle{annot} = [text width=4em, text centered]
    \tikzstyle{chemical}=[rectangle, rounded corners, fill=black!20] 
    \tikzstyle{cell}=[draw, rectangle, fill=orange!20, rounded corners,
        minimum width=5cm, minimum height=6cm]
    \tikzstyle{receptor}=[ellipse, fill=orange!30]
    \tikzstyle{nucleus}=[circle, minimum size=1.75cm, fill=my_red]
    \tikzstyle{organelles}=[chemical, fill=my_purple!70]


    \node[cell] (cell) at (5, -1) {};
    \node[cell, minimum width=4.8cm, minimum height=5.8cm, 
        fill=orange!10] (cell interior) at (5, -1) {};
    
    \node[chemical, fill=red!10, draw=red!30, line width=1pt] 
        (inputA) at (0, 0) {Signal A};
    \node[receptor, draw=orange!40, line width=.5pt] (receptor) at 
        (2.5, 0) {Receptor};

    \node[chemical, fill=red!10, draw=red!30, line width=1pt] 
        (inputB) at (0, -2) {Signal B};
    \node (fake receptor) at (3, -2) {};


    \node[nucleus, draw=my_red!130, line width=1.2pt] (nucleus) at 
        (6, 0.5) {Nucleus};
    \node[organelles, draw=my_purple, line width=1pt]  (organelles) at 
        (5, -3) {Organelles};
    \node[align=center] (other) at (5.7, -1.5) {Other cell\\components};

    \path (inputA) edge (receptor);
    \path (inputB) edge[-] node[above, xshift=-4pt] {\tiny by diffusion} 
        ($(fake receptor)+(0.1, 0)$);
    \path ($(fake receptor)$) edge[out=0, in=200] (nucleus);
    \path ($(fake receptor)$) edge[out=0, in=190] (other);
    \path ($(fake receptor)$) edge[out=0, in=160] (organelles);
    
    \path (receptor) edge[out=0, in=180] (nucleus);
    \path (receptor) edge[out=0, in=180] (other);
    \path (receptor) edge[out=0, in=180] (organelles);
    \end{tikzpicture}
\end{document}

