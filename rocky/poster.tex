\include{template}

\newcommand{\powerset}{\mathcal{P}}

%%%%%%%%%%%%%%%%%%%%%%%%%%%%%%%%%%%%%%%%%%%%%%%%%%%%%%%%%%%%%%%%%%%%%%%%%%%%%%%%%%%%%%
 
\title{\usebeamerfont*{title} Identification of cell signaling pathways 
    based on \\biochemical reaction kinetics repositories
}

\author{\vspace*{.5cm} \usebeamerfont*{author font} Gustavo Estrela de 
    Matos$^{\text{a, b, c}}$ \\ 
    \vspace*{.5cm}{\usebeamerfont*{supervisor font} Hugo Armelin, Marcelo S. Reis}}

\institute{%
    $^{\text{a}}$Institute of Mathematics and Statistics, University of São Paulo, Brazil\\%
$^{\text{b}}$Center of Toxins, Immune-response and Cell Signaling (CeTICS), Instituto Butantan, Brazil\\%
$^{\text{c}}$Special Laboratory of Cell Cycle, Instituto Butantan, Brazil}



%%%%%%%%%%%%%%%%%%%%%%%%%%%%%%%%%%%%%%%%%%%%%%%%%%%%%%%%%%%%%%%%%%%%%%%%%%%%%%%%%%%%%%
\begin{document}


\renewcommand{\figurename}{Fig.}

\begin{frame}
\begin{columns}

\leftcolumn{ 
\begin{block}{Cell Signaling Pathways}%
\paragraph{
Cell Signaling is a mechanism that allows the cell to change its 
    behaviour according to the environment.}
%(Fig.~\ref{fig:ucurve_example})
%\begin{figure}[h]
    %\begin{tabular}{l c r}
    %\centering
    %\subfigure {
        %\includegraphics[width=.2\textwidth,clip=true, trim={3cm 18cm 13cm 2cm}]{example_lattice_3.pdf}
    %} & \phantom{abcdefgh} &
    %\subfigure {
        %\includegraphics[width=.3\textwidth,clip=true, trim={1cm 0cm 1cm 2cm}]{example_lattice_chain_3.pdf}
    %}
    %\end{tabular}   
    %\captionsetup{width=.95\linewidth}
    %\caption{Exemplo de instância do problema U-curve com 3 características.}
    %\label{fig:ucurve_example}
%\end{figure}
\paragraph{
    A signal flows in a cell thorugh a cell signaling pathway.
}
\end{block}


%%%%%%%%%%%%%%%%%%%%%%%%%%%%%%%%%%%%%%%%%%%%%%%%%%%%%%%%%%%%%%%%%%%%%%%%                                     
\begin{block}{Identification of Signaling Pathways}%
\paragraph{}
\paragraph{}
\end{block}

%\vspace{-1cm}
\begin{block}{Feature Selection}
\paragraph{
}
\end{block}

%\vfill 
\vspace*{.5cm}%
\begin{block}{Acknowledgement}%
\vspace*{-1.5cm}%
\begin{figure}[h]
    \begin{tabular*}{0.7\textwidth}{c@{\extracolsep{\fill}}cc}
    \centering
    \subfigure {
        \includegraphics[clip=true, width=0.2\textwidth]{institutions/FAPESP.jpg}
    }
    &
    \subfigure {
        \includegraphics[clip=true, width=0.2\textwidth]{institutions/CNPq.png}
    }
    &
    \subfigure {
        \includegraphics[clip=true, width=0.1\textwidth]{institutions/capes.jpg}
    }
    \end{tabular*}   
\end{figure}
\vspace*{1.5cm}%
\end{block}%
} % end \leftcolumn



%%%%%%%%%%%%%%%%%%%%%%%%%%%%%%%%%%%%%%%%%%%%%%%%%%%%%%%%%%%%%%%%%%%%%%%%

\rightcolumn{              
\begin{block}{Bayesian Ranking of Models}%
%\leftfigparagraph{qrcode_featsel.png}{5}{10}{
%Usamos o {\em featsel}
%(\href{github.com/msreis/featsel}{github.com/msreis/featsel}), um 
%arcabouço em C++, para implementar e avaliar os novos algoritmos,
%comparando com soluções da literatura. Os testes foram conduzidos em uma
%servidora de 64 cores e 256 GB de memória RAM.
%} % paragraph
\end{block}


%%%%%%%%%%%%%%%%%%%%%%%%%%%%%%%%%%%%%%%%%%%%%%%%%%%%%%%%%%%%%%%%%%%
\begin{block}{Experiments}%
\paragraph{
}
\end{block}


%%%%%%%%%%%%%%%%%%%%%%%%%%%%%%%%%%%%%%%%%%%%%%%%%%%%%%%%%%%%%%%%%%%
\begin{block}{Conclusão}%
\paragraph{
}
\end{block}
}% end of right column
\end{columns}
\end{frame}
\end{document}
