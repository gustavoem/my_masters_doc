\documentclass[12pt]{article}
\usepackage[portuguese]{babel}
\usepackage[utf8]{inputenc}
\usepackage[usenames,dvipsnames]{color}
\usepackage{setspace}
\usepackage{amsmath}
\usepackage{amsfonts}
\usepackage{amssymb}
\usepackage{mathtools}
\usepackage[top=3cm, bottom=2cm, left=3cm, right=2cm]{geometry}
\usepackage{tikz}
\usepackage{textcomp}
\usepackage{float}
\usepackage[font={small,it}]{caption}
\usepackage{mhchem}
\title{Resumo de Projeto de Mestrado}

% packages added by Marcelo
\usepackage{lscape}    % for landscape pages
\usepackage{hyperref}  % to allow hyperlinks
\usepackage{booktabs}  % nicer table borders
\usepackage{subfigure} % add subfigures

% My default commands
\newcommand{\foreignword}[1]{\textit{#1}}
\newcommand{\toolname}[1]{\textit{#1}}
%\newcommand{\fieldR}{\mathbb{R}}
\newcommand{\powerset}{\mathcal{P}}
%\newcommand{\probability}{\mathbb{P}}
\newcommand{\expectation}{\mathbb{E}}
\newcommand{\algname}[1]{\texttt{#1}}
\newcommand{\langname}[1]{\texttt{#1}}
%\newcommand{\varname}[1]{\texttt{#1}}
%\newcommand{\floor}[1]{\lfloor #1 \rfloor}
%\newcommand{\ceil}[1]{\lceil #1 \rceil}
%\newcommand{\mathsc}[1]{{\normalfont\textsc{#1}}}
\newcommand{\forest}{\mathcal{F}}
\newcommand{\pfsnode}[1]{\mathbf #1}
\newcommand{\species}[1]{\textit{#1}}
\newcommand{\gender}[1]{\textit{#1}}

\graphicspath{{./img/}} 

\setstretch{1.5}

\begin{document}

% Double spacing
\doublespacing
\begin{flushleft}
{\LARGE Identificação de vias de sinalização celular baseada em repositórios de cinética de reações bioquímicas}

\bigskip
{\large {\bf Beneficiário:} \href{mailto:gustavo.estrela.matos@usp.br}{Gustavo Estrela de Matos}\\ 
        {\bf Responsável:} \href{mailto:marcelo.reis@butantan.gov.br}{Marcelo da Silva Reis}}\\
\end{flushleft}

% Rascunho desse documento
% Resumo entregue a FAPESP
% Resumo expandido
% 1. Vias de sinalização celular;
% 2. Modelos funcionais e identificação de vias de sinalização celular;
% 3. Como construir modelos funcionais; 
% 4. Neste sentido, apresentar o trabalho de Lulu Wu;
% 5. Entretanto, em seu trabalho...
% 6. Como propomos enfrentar as limitações do trabalho de Lulu Wu
%    - Extrair informações de mais repositórios
%    - Melhorar a busca pelo melhor modelo
%    - Construir uma métrica melhor de avaliação de modelos, que 
%      possua uma melhor penalização de overfitting.
% 7. Testaremos a metodologia com celulas...


\section*{Resumo do Projeto Proposto}
A construção de modelos funcionais é uma técnica comum para se 
estudar vias de sinalização celular e, quando a via estudada é pouco
conhecida, é possível que os modelos já propostos sejam incompletos, 
tornando necessário a sua modificação.
Lulu Wu apresentou em 2015, em sua dissertação de mestrado, um método 
para  modificar sistematicamente modelos funcionais, adicionando a estes
interações extraídas de repositórios como KEGG. Entretanto, esta 
metodologia apresentou limitações: a primeira é a incompletude do banco 
de dados de interações criado, que extraia informações apenas do 
repositório KEGG; a segunda, a falta de informações sobre constantes 
de velocidade de interações, que podem ser extraídas de repositórios 
como BioNumbers; a terceira, a dinâmica do algoritmo de busca, 
incremental, que pode não achar o mínimo global; e a última, a 
penalização na complexidade dos modelos, que era feita de maneira 
aleatória. Propomos neste trabalho enfrentar as limitações encontradas
pela metodologia de Lulu, criando um banco de dados de interações mais
completo e também novas funções de custo que sejam capazes de 
penalizar modelos mais complexos (como critério de informação Akaike e 
{\em Bayesian inference-based modeling}); esta penalização deve induzir,
em cadeias do  espaço de busca, curvas em u no custo dos modelos, 
portanto também propomos a criação de novos algoritmos de busca que 
explorem essa característica da função de custo. Por fim, esperamos 
testar nossa metodologia na identificação de vias de sinalização celular 
da linhagem tumoral murina Y1.

\end{document}
