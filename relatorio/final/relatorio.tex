\documentclass[12pt]{article}
\usepackage[portuguese]{babel}
\usepackage[utf8]{inputenc}
\usepackage[usenames,dvipsnames]{color}
\usepackage{setspace}
\usepackage{amsmath}
\usepackage{amsfonts}
\usepackage{amssymb}
\usepackage{mathtools}
\usepackage[top=3cm, bottom=2cm, left=3cm, right=2cm]{geometry}
\usepackage{tikz}
\usepackage{indentfirst}
\usepackage{textcomp}
\usepackage{float}
\usepackage[font={small,it}]{caption}
\usepackage{mhchem}
\title{Relatorio Final Mestrado}

% packages added by Marcelo
%
\usepackage{lscape}    % for landscape pages
\usepackage{hyperref}  % to allow hyperlinks
\usepackage{booktabs}  % nicer table borders
\usepackage{subfigure} % add subfigures

% My default commands
\newcommand{\foreignword}[1]{\textit{#1}}
\newcommand{\toolname}[1]{\textit{#1}}
%\newcommand{\fieldR}{\mathbb{R}}
\newcommand{\powerset}{\mathcal{P}}
%\newcommand{\probability}{\mathbb{P}}
\newcommand{\expectation}{\mathbb{E}}
\newcommand{\algname}[1]{\texttt{#1}}
\newcommand{\langname}[1]{\texttt{#1}}
%\newcommand{\varname}[1]{\texttt{#1}}
%\newcommand{\floor}[1]{\lfloor #1 \rfloor}
%\newcommand{\ceil}[1]{\lceil #1 \rceil}
%\newcommand{\mathsc}[1]{{\normalfont\textsc{#1}}}
\newcommand{\forest}{\mathcal{F}}
\newcommand{\pfsnode}[1]{\mathbf #1}
\newcommand{\species}[1]{\textit{#1}}
\newcommand{\gender}[1]{\textit{#1}}

\graphicspath{{./img/}} 

\setstretch{1.5}

\begin{document}

% FAPESP demands the usage of double spacing
%
\doublespacing

\begin{titlepage}
    \vfill 
    \begin{center}
        {\Large Relatório Científico Final -- Mestrado\\
         \bigskip
         Processo FAPESP 17/20575-9
        }
        
        \bigskip
        \bigskip
    
        {\LARGE Identificação de vias de sinalização celular baseada em
        repositórios de cinética de reações bioquímicas}

        \bigskip
        \bigskip
        {\Large {\bf Beneficiário:}
        \href{mailto:gustavo.estrela.matos@usp.br}{Gustavo Estrela de
        Matos}\\ 
        
        {\bf Responsável:}
        \href{mailto:marcelo.reis@butantan.gov.br}{Marcelo da Silva
        Reis}\\

        \bigskip
        \bigskip
        \bigskip
        \bigskip
        \bigskip
        \bigskip
        \bigskip
        Relatório referente aos trabalhos desenvolvidos entre 1 de
        janeiro de 2018 e 31 de dezembro de 2019

        \bigskip
        \bigskip
        \bigskip
        \bigskip
        \bigskip
        \bigskip
        \bigskip

        Laboratório Especial de Toxinologia Aplicada, Instituto
        Butantan\\
        \bigskip
        São Paulo, \today\\
        }

        \bigskip
        \bigskip

       

\end{center}
\end{titlepage}


\tableofcontents

\pagebreak



\section{Resumo do Projeto Proposto} \label{sec:resumo} % até 2 páginas
A construção de modelos funcionais é uma técnica comum para se 
estudar vias de sinalização celular e, quando a via estudada é pouco
conhecida, é possível que os modelos já propostos sejam incompletos, 
tornando necessário a sua modificação.
Lulu Wu apresentou em 2015, em sua dissertação de mestrado, um método 
para  modificar sistematicamente modelos funcionais, adicionando a estes
interações extraídas de repositórios como KEGG. Entretanto, esta 
metodologia apresentou limitações: a primeira é a incompletude do banco 
de dados de interações criado, que extraia informações apenas do 
repositório KEGG; a segunda, a falta de informações sobre constantes 
de velocidade de interações, que podem ser extraídas de repositórios 
como BioNumbers; a terceira, a dinâmica do algoritmo de busca, 
incremental, que pode não achar o mínimo global; e a última, a 
penalização na complexidade dos modelos, que era feita de maneira 
aleatória. Propomos neste trabalho enfrentar as limitações encontradas
pela metodologia de Lulu, criando um banco de dados de interações mais
completo e também novas funções de custo que sejam capazes de 
penalizar modelos mais complexos (como critério de informação Akaike e 
{\em Bayesian inference-based modeling}); esta penalização deve induzir,
em cadeias do  espaço de busca, curvas em u no custo dos modelos, 
portanto também propomos a criação de novos algoritmos de busca que 
explorem essa característica da função de custo. Por fim, esperamos 
testar nossa metodologia na identificação de vias de sinalização celular 
da linhagem tumoral murina Y1.


% - Estudo de seleção de modelos de cinética bioquímica
% - Estimação de verossimilhança marginal
% - Implementaçãoo do pacote SigNetMS
% - Alternativa de avaliação de modelos. (ABC-SysBio)
% - Avaliação de ambas alternativas
% - Paralelização do SigNetMS
% - Implementação eficiente de integração de sistemas de equações
%   diferenciais.
% - Teste com uma cadeia do espaço de busca.

\section{Atividades desenvolvidas}
% - Disciplinas cursadas

\subsection{Disciplinas cursadas}

\subsection{Estudo de seleção de modelos de via de sinalização celular}
O desenvolvimento deste projeto se iniciou com o estudo de métodos
capazes de avaliar a qualidade de um modelo de via de sinalização 
celular. Dado um conjunto de experimentos $\mathbf{D}$, que medem
concentrações de espécies químicas, precisamos escolher uma função de
custo $c (\mathbf{D}, M)$ que possa indicar a capacidade de um modelo
$M$ em reproduzir corretamente dados observados $\mathbf{D}$. O modelos
de via que utilizamos é definido por um conjunto de reações químicas,
produzindo um sistema de equações diferenciais, capaz de simular a
dinâmica das concentrações de espécies químicas da via ao longo do
tempo. Este sistema de equações diferenciais é criado utilizando leis de
cinética química, como no modelo de Michaelis-Menten, e possuem
constantes de velocidade que são usualmente desconhecidas; estas
constantes são parâmetros dos modelos de vias.

A função de custo escolhida deve considerar possíveis valores para as
constantes de velocidade do modelo avaliado. A abordagem de Lulu
Wu~\cite{Wu2015metodo}, por exemplo, utiliza um processo de simulated
annealing para encontrar o melhor conjunto de valores de parâmetros para
um modelo e conjunto de experimentos. Entretanto, esta abordagem teve
limitações que podem estar associadas a falta de informação a priori
sobre as constantes e também a falta de penalização apropriada a modelos
mais complexos. Por conta destas limitações, decidimos implementar uma
função de custo baseada em estatística Bayesiana, chamada de
verossimilhança marginal; denotamos $p (\mathbf{D} | M)$ a 
verossimilhança marginal de um conjunto de dados $\mathbf{D}$ dado um 
modelo $M$. Esta abordagem, apresentada no mesmo contexto no trabalho de 
Vyshemirsky e Girolami~\cite{Vyshemirsky2008}, permite a definição de 
informações a priori sobre constantes de velocidades e também induzem a 
penalização automática de modelos mais complexos.

% definir verossimilhança
% mostrar a integral e dizer que é difícil de ser implementada

Para calcular a verossimilhança marginal, precisamos definir a função de
verossimilhança, $p(\mathbf{D} | M, \theta)$, onde $\mathbf{D}$ é o 
conjunto de experimentos, $M$ é o modelo de interesse, e $\theta$ é um
conjunto de valores para os parâmetros (constantes de velocidade) do
modelo. Seguindo a abordagem de Kolch e Girolami, assumimos erro
Gaussiano e independente:
\begin{equation}
    p (\mathbf{D} | M, \theta) = \prod_{i = 1}^m
    p_{\mathcal{N}_{\left(0, \sigma^2\right)}} ([\phi (M,\theta) -
    \mathbf{D}]_i).
\label{eq:likelihood}
\end{equation}
Onde $\phi (M, \theta)$ é um vetor com os valores simulados de 
concentrações, em cada intervalo de tempo, pelo modelo $M$ usando 
parâmetros $\theta$. A partir da função de verossimilhança, podemos
obter a verossimilhança marginal com uma marginalização sobre os valores
de parâmetros de modelos, ou seja, integrando a função de 
verossimilhança sobre o espaço paramétrico, $\Theta$. Desta forma
podemos escrever:
\begin{equation}
    p (D | M) = \int_{\Theta} p (D | M, \theta) p (\theta | M)d\theta.
\label{eq:marginal_likelihood}
\end{equation}

Entretanto, a integral~\ref{eq:marginal_likelihood} normalmente não pode
ser calculada analiticamente. Para se calcular esta integral
analiticamente, seria necessário determinar a distribuição de 
probabilidade conjunta $p(D, \theta | M)$, o que não é possível
usualmente. Portanto, como é muito difícil (ou impossível) calcular a
verossimilhança marginal, utilizamos um estimador desse valor como nossa
função de custo. Este estimador é construído utilizando um método
conhecido como Integral Termodinâmica~\cite{Friel2008}.

\subsection{Estimação de verossimilhança marginal}

\subsection{Implementação do pacote SigNetMS}

\subsection{Alternativa de avaliação de modelos}
% dá pra falar do Pull Request que eu fiz aqui.

\subsection{Paralelização do SigNetMS}

\subsection{Implementação eficiente de integração de sistemas de
equações diferenciais}

\subsection{Testes da metodologia em uma cadeia do espaço de busca}





\addcontentsline{toc}{section}{Referências}
\bibliographystyle{unsrt} 
\bibliography{bib-relatorio}
\end{document}
