\documentclass[12pt]{article}
\usepackage[portuguese]{babel}
\usepackage[utf8]{inputenc}
\usepackage[usenames,dvipsnames]{color}
\usepackage{setspace}
\usepackage{amsmath}
\usepackage{amsfonts}
\usepackage{amssymb}
\usepackage{mathtools}
\usepackage[top=3cm, bottom=2cm, left=3cm, right=2cm]{geometry}
\usepackage{tikz}
\usepackage{indentfirst}
\usepackage{textcomp}
\usepackage{float}
\usepackage[font={small,it}]{caption}
\usepackage{mhchem}
\title{Relatorio Final Mestrado}

% packages added by Marcelo
%
\usepackage{lscape}    % for landscape pages
\usepackage{hyperref}  % to allow hyperlinks
\usepackage{booktabs}  % nicer table borders
\usepackage{subfigure} % add subfigures

% My default commands
\newcommand{\foreignword}[1]{\textit{#1}}
\newcommand{\toolname}[1]{\textit{#1}}
%\newcommand{\fieldR}{\mathbb{R}}
\newcommand{\powerset}{\mathcal{P}}
%\newcommand{\probability}{\mathbb{P}}
\newcommand{\expectation}{\mathbb{E}}
\newcommand{\algname}[1]{\texttt{#1}}
\newcommand{\langname}[1]{\texttt{#1}}
%\newcommand{\varname}[1]{\texttt{#1}}
%\newcommand{\floor}[1]{\lfloor #1 \rfloor}
%\newcommand{\ceil}[1]{\lceil #1 \rceil}
%\newcommand{\mathsc}[1]{{\normalfont\textsc{#1}}}
\newcommand{\forest}{\mathcal{F}}
\newcommand{\pfsnode}[1]{\mathbf #1}
\newcommand{\species}[1]{\textit{#1}}
\newcommand{\gender}[1]{\textit{#1}}

\graphicspath{{./img/}} 

\setstretch{1.5}

\begin{document}

% FAPESP demands the usage of double spacing
%
\doublespacing

\begin{titlepage}
    \vfill 
    \begin{center}
        {\Large Relatório Científico Final -- Mestrado\\
         \bigskip
         Processo FAPESP 17/20575-9
        }
        
        \bigskip
        \bigskip
    
        {\LARGE Identificação de vias de sinalização celular baseada em
        repositórios de cinética de reações bioquímicas}

        \bigskip
        \bigskip
        {\Large {\bf Beneficiário:}
        \href{mailto:gustavo.estrela.matos@usp.br}{Gustavo Estrela de
        Matos}\\ 
        
        {\bf Responsável:}
        \href{mailto:marcelo.reis@butantan.gov.br}{Marcelo da Silva
        Reis}\\

        \bigskip
        \bigskip
        \bigskip
        \bigskip
        \bigskip
        \bigskip
        \bigskip
        Relatório referente aos trabalhos desenvolvidos entre 10 de
        dezembro de 2018 e 31 de dezembro de 2019

        \bigskip
        \bigskip
        \bigskip
        \bigskip
        \bigskip
        \bigskip
        \bigskip

        Laboratório Especial de Toxinologia Aplicada, Instituto
        Butantan\\
        \bigskip
        São Paulo, \today\\
        }

        \bigskip
        \bigskip

       

\end{center}
\end{titlepage}


\tableofcontents

\pagebreak



\section{Resumo do Projeto Proposto} \label{sec:resumo} % até 2 páginas
A construção de modelos funcionais é uma técnica comum para se 
estudar vias de sinalização celular e, quando a via estudada é pouco
conhecida, é possível que os modelos já propostos sejam incompletos, 
tornando necessário a sua modificação.
Lulu Wu apresentou em 2015, em sua dissertação de mestrado, um método 
para  modificar sistematicamente modelos funcionais, adicionando a estes
interações extraídas de repositórios como KEGG. Entretanto, esta 
metodologia apresentou limitações: a primeira é a incompletude do banco 
de dados de interações criado, que extraia informações apenas do 
repositório KEGG; a segunda, a falta de informações sobre constantes 
de velocidade de interações, que podem ser extraídas de repositórios 
como BioNumbers; a terceira, a dinâmica do algoritmo de busca, 
incremental, que pode não achar o mínimo global; e a última, a 
penalização na complexidade dos modelos, que era feita de maneira 
aleatória. Propomos neste trabalho enfrentar as limitações encontradas
pela metodologia de Lulu, criando um banco de dados de interações mais
completo e também novas funções de custo que sejam capazes de 
penalizar modelos mais complexos (como critério de informação Akaike e 
{\em Bayesian inference-based modeling}); esta penalização deve induzir,
em cadeias do  espaço de busca, curvas em u no custo dos modelos, 
portanto também propomos a criação de novos algoritmos de busca que 
explorem essa característica da função de custo. Por fim, esperamos 
testar nossa metodologia na identificação de vias de sinalização celular 
da linhagem tumoral murina Y1.


% - Estudo de seleção de modelos de cinética bioquímica
% - Estimação de verossimilhança marginal
% - Implementaçãoo do pacote SigNetMS
% - Alternativa de avaliação de modelos. (ABC-SysBio)
% - Avaliação de ambas alternativas
% - Paralelização do SigNetMS
% - Implementação eficiente de integração de sistemas de equações
%   diferenciais.
% - Teste com uma cadeia do espaço de busca.

\section{Atividades desenvolvidas}
% - Disciplinas cursadas

\subsection{Disciplinas cursadas}

\subsection{Resumo de atividades anteriores}
\subsubsection{Estudo de seleção de modelos de via de sinalização celular}
O desenvolvimento deste projeto se iniciou com o estudo de métodos
capazes de avaliar a qualidade de um modelo de via de sinalização 
celular. Dado um conjunto de experimentos $\mathbf{D}$, que medem
concentrações de espécies químicas, precisamos escolher uma função de
custo $c (\mathbf{D}, M)$ que possa indicar a capacidade de um modelo
$M$ em reproduzir corretamente dados observados $\mathbf{D}$. O modelos
de via que utilizamos é definido por um conjunto de reações químicas,
produzindo um sistema de equações diferenciais, capaz de simular a
dinâmica das concentrações de espécies químicas da via ao longo do
tempo. Este sistema de equações diferenciais é criado utilizando leis de
cinética química, como no modelo de Michaelis-Menten, e possuem
constantes de velocidade que são usualmente desconhecidas; estas
constantes são parâmetros dos modelos de vias.

A função de custo escolhida deve considerar possíveis valores para as
constantes de velocidade do modelo avaliado. A abordagem de Lulu
Wu~\cite{Wu2015metodo}, por exemplo, utiliza um processo de simulated
annealing para encontrar o melhor conjunto de valores de parâmetros para
um modelo e conjunto de experimentos. Entretanto, esta abordagem teve
limitações que podem estar associadas a falta de informação a priori
sobre as constantes e também a falta de penalização apropriada a modelos
mais complexos. Por conta destas limitações, decidimos implementar uma
função de custo baseada em estatística Bayesiana, chamada de
verossimilhança marginal; denotamos $p (\mathbf{D} | M)$ a 
verossimilhança marginal de um conjunto de dados $\mathbf{D}$ dado um 
modelo $M$. Esta abordagem, apresentada no mesmo contexto no trabalho de 
Vyshemirsky e Girolami~\cite{Vyshemirsky2008}, permite a definição de 
informações a priori sobre constantes de velocidades e também induzem a 
penalização automática de modelos mais complexos.

% definir verossimilhança
% mostrar a integral e dizer que é difícil de ser implementada

Para calcular a verossimilhança marginal, precisamos definir a função de
verossimilhança, $p(\mathbf{D} | M, \theta)$, onde $\mathbf{D}$ é o 
conjunto de experimentos, $M$ é o modelo de interesse, e $\theta$ é um
conjunto de valores para os parâmetros (constantes de velocidade) do
modelo. Seguindo a abordagem de Kolch e Girolami, assumimos erro
Gaussiano e independente:
\begin{equation}
    p (\mathbf{D} | M, \theta) = \prod_{i = 1}^m
    p_{\mathcal{N}_{\left(0, \sigma^2\right)}} ([\phi (M,\theta) -
    \mathbf{D}]_i).
\label{eq:likelihood}
\end{equation}
Onde $\phi (M, \theta)$ é um vetor com os valores simulados de 
concentrações, em cada intervalo de tempo, pelo modelo $M$ usando 
parâmetros $\theta$. A partir da função de verossimilhança, podemos
obter a verossimilhança marginal com uma marginalização sobre os valores
de parâmetros de modelos, ou seja, integrando a função de 
verossimilhança sobre o espaço paramétrico, $\Theta$. Desta forma
podemos escrever:
\begin{equation}
    p (\mathbf{D} | M) = \int_{\Theta} p (\mathbf{D} | M, \theta) p
    (\theta | M)d\theta.
\label{eq:marginal_likelihood}
\end{equation}

Entretanto, a integral~\ref{eq:marginal_likelihood} normalmente não pode
ser calculada analiticamente. Para se calcular esta integral
analiticamente, seria necessário determinar a distribuição de 
probabilidade conjunta $p(D, \theta | M)$, o que não é possível
usualmente. Portanto, como é muito difícil (ou impossível) calcular a
verossimilhança marginal, utilizamos um estimador desse valor como 
função de custo. Este estimador é construído utilizando um método
conhecido como Integral Termodinâmica~\cite{Friel2008}.

\subsubsection{Estimação de verossimilhança marginal}
% - É possível reescrever a verossimilhança marginal em outra integral 
% usando potencias de posteriori.
% - Definir o que é a potência de posteriori
% - Dizer que é possível provar que log p (D | M) = ...
% - Dizer que é possível estimar esse valor de diferentes maneiras, mas
% todos dependem de criar uma amostra das power posteriors
O trabalho de Friel et al. ~\cite{Friel2008} mostra que é possível
reescrever o logaritmo da integral~\ref{eq:marginal_likelihood} como uma 
outra integral, um pouco menos simples, mas que nos permite criar 
estimadores para o logaritmo da verossimilhança marginal. Esta segunda 
forma de se escrever a verossimilhança marginal é baseada na integração 
de várias distribuições de probabilidade que são intermediárias entre as 
distribuições a priori e a posteriori das constantes de velocidade. As 
distribuições intermediárias são denominadas potências de posteriori.

Dada uma distribuição a priori $p (\theta | M)$ e a posteriori $p
(\theta | \mathbf{D}, M)$, definimos a distribuição potência de
posteriori como:
\begin{equation*}
    p_{\beta} (\theta) = \frac{p (\mathbf{D} | \theta, M)^\beta 
        p(\theta | M)}{z (\beta)},
\end{equation*}
onde
\begin{equation*}
    z (\beta) = \int_\Theta p (\mathbf{D} | \theta, M)^\beta 
        p(\theta | M) d\theta.
\end{equation*}
Note que $p_{0} (\theta)$ é a distribuição a priori e que $p_{1}
(\theta)$ é a distribuição a posteriori. Portanto, podemos dizer que
quando variamos o valor de $\beta$ entre 0 e 1 estamos produzindo
distribuição intermediárias que conectam a priori a posteriori. Friel et
al. provam que é possível escrever:
\begin{align}
    \int_0^1 \expectation_{p_\beta (\theta)} 
        [\ln p(\mathbf{D}|\theta, M)]d\beta 
    &= \int_0^1 \frac{d}{d\beta} \ln z(\beta) d\beta \notag \\
    &= \Big[\ln z(\beta)\Big]\bigg\rvert^1_0 \notag \\
    &= \ln p (\mathbf{D} | M).
    \label{eq:thermodynamic_integral}
\end{align}

O lado esquerdo da equação~\ref{eq:thermodynamic_integral} recebe o nome 
de integral termodinâmica. Este nome se justifica pela variação do
parâmetro $\beta$, que pode ser visto como um parâmetro de temperatura
nas distribuições potência de posteriori. Esta integral pode ser
estimada ou aproximada numericamente, permitindo acessar um valor 
próximo ao logaritmo da verossimilhança marginal. Para estimar ou
aproximar esta integral, é necessário construir amostras de
distribuições potência de posteriori para um conjunto finito de valores
de $\beta$.

Em resumo, reescrevemos o logaritmo da verossimilhança marginal como uma
integral que chamamos de integral termodinâmica. Esta integral pode ser
aproximada numericamente ou estimada. Para ambas opções, é necessário 
escolher uma sequência de valores para $\beta$ entre 0 e 1, e gerar
amostras das distribuições potência de posteriori para os respectivos
valores de $\beta$ escolhidos.

\subsubsection{Implementação do pacote SigNetMS}
% - explicar como decidimos fazer a amostragem das power posteriors
%   - etapa 1: burn-in
%   - etapa 2: burn-in informado
%   - etapa 3: amostragem
% - dizer qual tipo de estimador decidimos escolher
Após nossos estudos sobre seleção de modelos, decidimos implementar
um pacote Python que nos providenciaria uma aproximação do logaritmo da
verossimilhança marginal, usando os conceitos de integral termodinâmica.
Este pacote foi implementado e recebeu o nome SigNetMS, e está 
disponível em um repositório público no
\href{https://github.com/gustavoem/SigNetMS}{GitHub}
\footnote{https://github.com/gustavoem/SigNetMS}. O pacote SigNetMS 
recebe como entrada um arquivo no formato \emph{Systems Biology Makup
Language} (SBML)~\cite{Hucka2003}, com a definição das reações e 
constantes de velocidade da via; um arquivo \emph{Extensible Markup 
Language} (XML) com resultados de experimentos; e um arquivo XML com 
definições de distribuições a priori para constantes de velocidade das
reações da via. O pacote pode devolver como resposta o valor aproximado
de $\log p (\mathbf{D} | M)$ e também amostras das distribuições
potência de posteriori.

O pacote SigNetMS é capaz de processar modelos no formato SBML e criar
os correspondentes sistemas de equações diferenciais ordinárias.
Utilizando o pacote \algname{Scipy} e seu integrador \algname{odeint} é
possível integrar esses sistema de equações diferenciais, criando uma
simulação da dinâmica das concentrações gerada pelo par modelo e 
parâmetros $(M, \theta)$. Esta simulação é utilizada na função de 
verossimilhança, implementada de acordo com a 
equação~\ref{eq:likelihood}. A verossimilhança do experimento, dado um
modelo e conjunto de parâmetros, é usada no processo de geração da
amostra de distribuições potência de posteriori e também na aproximação 
da verossimilhança marginal.

Para calcular a verossimilhança marginal, o pacote SigNetMS segue uma 
abordagem que faz uma aproximação numérica da 
integral~\ref{eq:thermodynamic_integral}. Esta aproximação é 
simplesmente a aplicação da regra dos trapézios para integrais. Desta 
maneira, é necessário escolher uma sequência de valores para $\beta$, 
o que também determina o conjunto de potências de posteriori que serão 
amostradas. O pacote SigNetMS faz esta escolha de $\beta_1, \ldots, 
\beta_T$ da maneira recomendada por Friel et al.:
\begin{equation*}
    \beta_t = \left(\frac{t - 1}{T - 1}\right)^{c}, 
\end{equation*}
com $T = 20$ e $c = 5$. Assim, aplicando a regra dos trapézios na
integral~\ref{eq:thermodynamic_integral}, podemos escrever:
\begin{equation*}
    \log{p(\mathbf{D}| M)} \approx \sum_{t = 0}^{T - 1} 
        (\beta_{t + 1} - \beta_t)
    \frac{
    \expectation_{p_{\beta_{t + 1}} (\theta)}[\log p(D | M, \theta)]
+ 
    \expectation_{p_{\beta_{t}} (\theta)}[\log p(D | M, \theta)]}
{2}
\end{equation*}
Além disso, se considerarmos que a potência de posteriori $\beta_t$ tem
$M_t$ parâmetros amostrados, então podemos substituir a esperança por
um estimador de seu valor, produzindo a equação:
\begin{equation}
\log{p(D| M)} \approx \sum_{t = 0}^{T - 1} (\beta_{t + 1} - \beta_t)
\frac{
    \frac{1}{M_{t + 1}}
    \sum_{i = 1}^{M_{t + 1}}  \log p(D | M, \theta^{(t + 1, i)})
+ 
    \frac{1}{M_t}
    \sum_{i = 1}^{M_t}  \log p(D | M, \theta^{(t, i)})}
{2}
\end{equation}
onde $\theta^{(j, i)}$ é o i-ésimo parâmetro amostrado para a
potência de posteriori $p_{\beta_j}(\theta)$. Resta agora definir como
as amostras de potência de posteriori são criadas.

As amostras de potência de posteriori são criadas em três etapas que
utilizam o algoritmo Metropolis-Hastings. Esse algoritmo permite gerar
uma amostra de uma distribuição (geralmente desconhecida ou difícil de
se amostrar) a partir de uma distribuição de proposta, com a criação de
uma cadeia de Markov. Chamamos estas três etapas de burn-in, burn-in
informativo e amostragem final; todas estas etapas utilizam a
distribuição log-normal como distribuição de proposta para os 
parâmetros. 

Na versão do SigNetMS que utilizamos até a escrita do relatório parcial,
a etapa de burn-in amostrava a distribuição a posteriori (ou seja,
apenas uma cadeia) de parâmetros de maneira independente, com uma
distribuição de pulo com covariância diagonal. Na etapa de burn-in
informativo, uma amostragem similar a primeira etapa ocorria, porém
utilizando uma distribuição de pulo com matriz de covariância diagonal
tal que cada variância fosse igual a variância da amostra atual. Por
fim, na ultima etapa, $T$ cadeias eram geradas, uma para cada potência
de posteriori escolhida, com distribuição de pulo igual a última
utilizada na etapa anterior.

\subsubsection{Primeiros testes da metodologia}
Ainda no primeiro ano do projeto, testamos o SigNetMS na seleção de 
modelos. Porém, os resultados eram satisfatórios apenas para exemplos
pequenos. Com exemplos maiores e com mais parâmetros, o pacote não
apresentava bons resultados. Por esse motivo, começamos o segundo
período do projeto (entre dezembro de 2018 e dezembro de 2019) ajustando
nossa implementação 

\subsection{Melhorando a estimação da posteriori}
% Falar sobre a autocorrelação das amostras quando usávamos a
% metodologia anterior.

% Usando multiplas cadeias em todas etapas diminuimos a auto-correlação

% Falar sobre usar a matriz de covariância para o MH ter informação 

% TODO: elaborar isso aqui um pouco mais...

Logo após a entrega do relatório parcial, identificamos que as amostras
de potência de posteriori geradas pelo SigNetMS eram muito parecidas.
Isso indicava que existia uma correlação grande entre as cadeias
amostradas. Consultando o trabalho de Xu et al.~\cite{Xu2010} e Friel
et al.~\cite{Friel2008} identificamos que as duas primeiras etapas de
amostragem, burn-in e burn-in informativo também deveriam ser feitas
para cada potência de posteriori escolhida, e não apenas para a
distribuição a posteriori. Além disso, identificamos que a distribuição
de pulo da etapa de burn-in informativo poderia ter como covariância a
covariância amostral do conjunto de parâmetros aceitos até o instante. 

\subsection{Alternativa de avaliação de modelos}
% dá pra falar do Pull Request que eu fiz aqui.
Ao mesmo tempo que investigamos possíveis erros na metodologia do
SigNetMS também experimentamos uma outra função de custo Bayesiana para
seleção de modelos de via, chamada ABC-SMC~\cite{Liepe2014}. A função
ABC-SMC se baseia em um método de geração de amostras conhecido como
\foreignword{Approximate Bayesian Computation} (ABC). Na função de custo
ABC-SMC, amostras da distribuição $p(\theta, M | \mathbf{D})$ são 
geradas, permitindo estimar o valor de $p (M | \mathbf{D})$.

Podemos escrever um algoritmo genérico ABC que se propõe a gerar 
amostras da distribuição $p(\theta, M | \mathbf{D})$ com os seguintes 
passos:
\begin{enumerate}
    \item Amostre um parâmetro candidato $(\theta^* | M^*)$ de uma
        distribuição de propostas.
    \item Simule o par $(M^*, \theta^*)$ com os mesmos intervalos de
        tempo e para a mesma métrica do experimento $\mathbf{D}$, 
        gerando $\mathbf{D}^*$.
    \item Calcule, para alguma métrica de distância $d$, se o valor $d 
        (\mathbf{D}, \mathbf{D^*})$ for menor que um $\epsilon$
        pré-determinado, então adicione o par $(\theta^*, M^*)$ a 
        amostra.
    \item Repita até uma condição de parada.
\end{enumerate}

\subsection{Paralelização do SigNetMS}

\subsection{Implementação eficiente de integração de sistemas de
equações diferenciais}

\subsection{Testes da metodologia em uma cadeia do espaço de busca}





\addcontentsline{toc}{section}{Referências}
\bibliographystyle{unsrt} 
\bibliography{bib-relatorio}
\end{document}
