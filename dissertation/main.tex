\documentclass[12pt, twoside]{report}
\usepackage[utf8]{inputenc}
\usepackage[a4paper, top=30mm, left=20mm, bottom=20mm,
    right=20mm]{geometry}
\usepackage[dvipsnames]{xcolor}
\usepackage{graphicx}
\usepackage{color}
\usepackage{fancyhdr}
\fancyhead[LO,RE]{\itshape \nouppercase Chapter \arabic{chapter}}
\usepackage{amssymb}
\usepackage{csquotes}
\usepackage{amsmath}
\usepackage{amsthm}
\usepackage{mathtools} % shortintertext
\usepackage{faktor}
\usepackage[backend=bibtex, 
            maxbibnames=10,
            style=alphabetic]{biblatex}
%\renewcommand{\familydefault}{\sfdefault}
%\usepackage{helvet}
\usepackage{hyperref}
\usepackage{subfigure}
\usepackage[font={small,it}]{caption}

\usepackage{algpseudocode}
\usepackage[plain]{algorithm}
\usepackage[shortlabels]{enumitem}
\usepackage{booktabs}
\usepackage{mhchem} % chemical reactions
\usepackage{cleveref} % multireference (eq 1.2-1.5)
\usepackage{pgfgantt} % nice Gantt diagrams
\usepackage{bm}

% teste

\pagestyle{fancy}

% Images path
\graphicspath{ {img/} }

% ABNT foreign words should be in italic
\newcommand{\foreignword}[1]{\textit{#1}}
\newcommand{\toolname}[1]{\textit{#1}}
\newcommand{\fieldR}{\mathbb{R}}
\newcommand{\naturals}{\mathbb{N}}
\newcommand{\powerset}{\mathcal{P}}
\newcommand{\probability}{\mathbb{P}}
\newcommand{\expectation}{\mathbb{E}}
\newcommand{\algname}[1]{\texttt{#1}}
\newcommand{\langname}[1]{\texttt{#1}}
\newcommand{\varname}[1]{\texttt{#1}}
\newcommand{\floor}[1]{\lfloor #1 \rfloor}
\newcommand{\ceil}[1]{\lceil #1 \rceil}
\newcommand{\mathsc}[1]{{\normalfont\textsc{#1}}}
\newcommand{\forest}{\mathcal{F}}
\newcommand{\pfsnode}[1]{\mathbf #1}
\newcommand{\species}[1]{\textit{#1}}
\newcommand{\gender}[1]{\textit{#1}}

% remove returns of the same line in pseudocodes
\algrenewcommand\Return{\State \algorithmicreturn{} } 

\DeclareMathOperator*{\argmin}{argmin} 

\addbibresource{references.bib}

\newtheorem{mydefinition}{Definition}
\numberwithin{mydefinition}{section}
\newtheorem{mytheorem}{Theorem}
\numberwithin{mytheorem}{section}
\newtheorem{mylemma}{Lemma}
\numberwithin{mylemma}{section}
\newtheorem{corollary}{Corollary}
\numberwithin{corollary}{section}


\begin{document}
\pagenumbering{roman}
\thispagestyle{empty}
\begin{center}
{\Large
{\bf Identification of cell signaling pathways based on biochemical 
    reaction kinetics repositories}\\
\bigskip
\bigskip
\bigskip
\bigskip
    {\bf \href{mailto:gustavo.estrela.matos@gmail.com}{Gustavo Estrela de Matos}}\\
\bigskip
\bigskip
\bigskip
\bigskip
\textsc{
    Text presented\\[-0.25cm] 
    to\\[-0.25cm]
    Institute of Mathematics and Statistics\\[-0.25cm]
    of the\\[-0.25cm]
    University of São Paulo\\[-0.25cm]
    For\\[-0.25cm]
    The Qualification Exam Of Master Of Science\\
    }
\bigskip
\bigskip
\bigskip
\bigskip
Field of knowledge: Computer Science\\
\bigskip
Advisor: Dr. Marcelo da Silva Reis\\
\bigskip
\bigskip
\bigskip
\bigskip
\bigskip
\bigskip
\bigskip
\bigskip
Center of Toxins, Immune-Response and Cell Signaling (CeTICS)\\
\bigskip
Special Laboratory of Cell Cycle, Butantan Institute\\
\bigskip
\bigskip
{\normalsize During the development of this work the author received 
    financial support from FAPESP.}\\
\bigskip
\bigskip
\bigskip
São Paulo, \today
}
\end{center}
\newpage

\chapter*{Abstract}
Cell signaling pathways are composed of a set of biochemical reactions 
that are associated with signal transmission within the cell and its 
surroundings. Traditionally, these pathways are identified through 
statistical analyses on results from biological assays, in which 
involved chemical species are quantified. However, once generally it is 
measured only a few time points for a fraction of the chemical species, 
to effectively tackle this problem it is required to design and simulate 
functional dynamic models. Recently, it was introduced a method to 
design functional models, which is based on systematic modifications of 
an initial model through the inclusion of biochemical reactions, which 
in turn were obtained from the interactome repository KEGG. 
Nevertheless, this method presents some shortcomings that impair the 
estimated model; among them are the incompleteness of the information 
extracted from KEGG, the absence of rate constants, the usage of 
sub-optimal search algorithms and an unsatisfactory overfitting 
penalization. In this project, we propose a new methodology for 
identification of cell signaling pathways, which will make use of a 
myriad of public interactome and biochemical reaction kinetics 
repositories to deal with the incompleteness of a priori information. 
Moreover, we will use optimal algorithms for model selection, as well as 
more effective cost functions for overfitting penalization. The new 
methodology will be tested on artificial instances and also on cell 
signaling pathways identification in our case study, the Y1 mouse 
adrenocortical tumor cell line. (AU)
 
\tableofcontents

\clearpage
\pagenumbering{arabic} 

\nocite{*}
\chapter{Introduction}
\label{chap:intro}
%begin-include

% Outline of this section
% First we should start with a two-page description of the project with:
% - What are cell signaling pathways
% - What types of computational models we have for these pathways and 
% which one do we use
% - What can we do with those networks and how hard is it to get one to 
% work
% - What are the basic tasks to build a pathway
% - What has Lulu done and what are the limitations of her work
% - How do we intend to surmount those limitations


% Subsection Then we should state the main objectives/challenges of our 
% work
% Subsection Then we should give a short description of the work 


% What are cell signaling pathways and how is it important to study them
% - control how cell behaves in different types of environments
% - cancer cells have bad behavior
% - that's one reason to study signaling networks
Cell signaling pathways can be seen as cascades of chemical interactions 
that allow the communication between the cell environment and the 
cell itself. These pathways are also able to regulate many cell 
functions, including DNA replication, cell division and death. We
can observe the functioning of signaling pathways as a mechanism that 
can conform the cell behavior with signals that come from the 
environment conditions in which the cell is placed. The studies of cell 
signaling pathways can lead to determining how cells can respond to 
different stimuli; for instance, with the studies of signaling pathways
activated by a chemical species, one could determine how an unhealthy 
cell would respond to a drug containing this species.

% There are computational models for signaling networks
% - Michaelis Menten equations for chemical interactions
% - With a system of ODES we can then simulate the cell behavior
% - However, the huge number of interactions happening in the cell makes
%   it impossible to consider everything.
% - Therefore we must know 
It is possible to construct mathematical models to represent a set of
chemical reactions and consequently a signaling pathway. One approach on 
the modeling of those interactions is based on the law of mass action. 
This law proposes that the rate of a chemical reaction is proportional 
to the product of reactant concentrations, i.e., we can calculate the 
concentration change rate of a species in an interaction by calculating 
the product of reactant concentrations up to a multiplying constant. 
If we consider the set of interactions of a signaling pathway, we can
then come up with a system of ordinary differential equations (ODEs) 
that can model the dynamics of the concentration of each chemical 
species from the pathway. Generally, those systems are complex and 
cumbersome, if not impossible, to be solved analytically; therefore, we 
resort on computational tools that apply numerical methods to 
approximate solutions of these systems.

% How do we create these computational models and what should they do?
In this work, we are interested in computational models that can 
reproduce the behavior of signaling pathways, comparing simulations
generated by those models to experimental measures, generally based on
an analytical technique for protein detection called 
Western blot. Figure~\ref{fig:signal_pathway_example} shows a
set of interactions as well as parameters of a model of a signaling 
pathway. To create computational models that are able to simulate the 
behavior of a signaling pathway, two main tasks need to be 
accomplished, which will be described in the following.

\begin{figure}[!ht]
\centering 
    \includegraphics[width=\textwidth]{introduction/csp_example.pdf}
\caption{The above diagram show a hypothesis for a signaling pathway 
    that flows through Raf-MEK-ERK cascade. Names in bold represent 
    chemical species. Names in italic represent parameters of the 
    ordinary differential equation of each interaction.Horizontal arrows
    represent phosphorylation when directed from left to right or 
    dephosphorylation when directed in the opposite direction. Other 
    arrows represent positive feedback if they are directed downwards or 
    negative feedback otherwise. Original image of Marcelo S. Reis et
    al. (2017)~\cite{Reis2017}.}
\label{fig:signal_pathway_example}
\end{figure}

The first task one must complete to create a model is to determine a set 
of interactions that will be considered in the ODE system. Searching for 
pathway maps on the Kyoto Encyclopedia of Genes and Genomes 
(KEGG)~\cite{Kanehisa2000kegg} is a good start for this task. The KEGG 
PATHWAY Database provides manually drawn diagrams that represent 
signaling pathways created with experimental evidences. However, it is 
possible that there is no pathway on KEGG that is able to correctly 
represent the biological experiment of interest; for those situations, 
it is necessary to modify the pathway by adding or removing 
interactions. One might reason that we should use as many interactions 
as we can to get a better simulation. Indeed, a model that consider more
reactions is more general and might be able to reproduce different sets
of experimental data. However, being more general may imply in poor or 
computationally unfeasible models due to two reasons: first, 
complex models will require more time for a numerical solution 
computation, which may be unfeasible due to limited computational 
resources; and second, when considering many interactions, we are also 
considering many parameters (constants of the differential equations 
system), and finding appropriate values for them becomes harder as we 
increase the number of parameters.

The second task is to find values for all the model parameters. We can
highlight two approaches for this task; one can either fetch values for 
these constants from the literature, or find values that make 
the model output approximate the experimental observations. For the
first approach, repositories such as BioModels~\cite{le2006biomodels} 
can be used; for the second, statistical and optimization
methods are needed. For optimization, it is necessary to define a metric
that can evaluate how close a set of parameters brings the simulation to 
the experimental observation. Only after that, it is possible to search 
for the optimal parameter in the parameter space. Statistical inference, 
on the other hand, will usually try to maximize some likelihood function 
(find parameters that makes the data more likely to happen) on a more 
classical approach, while in a Bayesian approach the goal is usually to
compute some posterior distribution for the model parameters (the 
probability of parameter values given the experimental observation).


% However, we might have missed some interactions or even added
% irrelevant ones
After completing both tasks, however, as we mentioned before, we might 
still not have found a pair of model and parameter values that fairly 
approximates the biological experiment of interest. That could indicate 
that the set of chemical interactions chosen for the model is 
incomplete or has interactions that are not relevant for the biological 
experiment. Therefore, it is desirable to construct a systematic method 
of modifying the set of chemical reactions of the model in order to 
find a good set to represent the signaling pathway.

% Lulu solved it as a combinatorial problem
With the title ``A method to modify molecular signaling pathways through
examination of interactome databases"~\cite{Wu15}, Lulu Wu presented in 
her masters dissertation a methodology to systematically modify 
computational models of signaling pathways to better simulate biological 
experiments. Starting with a model that does not approximate well the 
biological data, this methodology proposes to include into the model
a set of chemical interactions that are relevant for the biochemical 
experiment, and consequently approximate the model simulation to 
experimental data. This set of interactions is a subset of interactions 
from a database created by Wu, joining information from many static 
maps of signaling pathways available on KEGG. The choice of this subset 
can be modeled as a combinatorial optimization problem, the feature 
selection problem, in which the search space is the set of all possible 
subsets of interactions (features) to be added. The cost function of 
this problem, however, is not as simple to define as the search space. 
Note that points from the search space do not fully define models, 
because there is still need to define parameter values to produce a 
simulation. Therefore, to analyze the quality of a model, the cost 
function must take into account the set of values for the model 
parameters. As an example, we could define the cost as the minimum 
distance between experimental and model measures considering all 
possibilities of parameter values; however, unfortunately, finding parameter values that minimizes such cost is a NP-hard problem.

Once this is a hard problem, the method presented by Wu 
implements a heuristic version of this cost function; moreover, the 
algorithm used to traverse the search space is also a heuristics. The 
cost function heuristics is based on a simulated annealing procedure 
that searches for a set of parameter values trying to minimize (as much 
as possible) the distance between model and experimental measures. The 
best found distance times $-1$ is then considered as the cost of the 
model. The size of the search space is the number of all 
possible subsets of interactions to be added, and this number grows
exponentially on the number of interactions from the database. That
explains the need of a heuristic to traverse the search space. This
heuristic is based on the greedy algorithm called Sequential Forward
Selection (SFS)~\cite{Whitney1971}. The heuristic implemented by Wu
selects a fixed number of interactions from the database and then
creates candidate models by adding to the current solution the
respective interaction; then, after evaluating the cost function for
each model, the algorithm moves to the best candidate.

% 1. incompleteness of database (we can use more than KEGG)
% 2. no information about reactions constants
% 3. the algorithm can only add interactions but not remove
% 4. cost function apply penalization randomly and arbitrarily
The results presented on Wu's dissertation show that the method is 
useful when there are only a few differences between the starting model
and a model that closely approximates the biological experiment. This
limitation could be explained by the intrinsic difficulty of the 
problem, which demands fitting complex models with few experimental 
data; however, we would like to highlight three aspects of the work that 
contributes to its limitations. The first aspect is that the constructed 
database could be more nearly complete, adding information from other 
interactome databases, such as STRING~\cite{Szklarczyk2010}, and also by 
adding information about model parameters, i.e. chemical reaction 
constants, that are available in other databases, e.g. the 
SABIO-RK~\cite{Wittig2011} database. Second, the search algorithm used
to modify the models can only add interactions, therefore, if the 
algorithm starts with (or add along the search) a spurious interaction 
on the topology, then the algorithm will not be able to ``regret" that 
interaction even though there might be similar solutions without it with a
better fit. Third, the cost function does not include a proper 
penalization of complex models; the used penalization is based on a 
execution time limit on the simulated annealing procedure, implying on a 
random penalization for more complex models, which typically demand more 
execution time. Without a proper penalization, the algorithm is doomed 
to select overly complex models that, even with a good fit to the 
experimental data, are not likely to reproduce the same experiment
conduced with any kind of perturbation to the biological environment or 
to the data collected.

Despite the limitations showed in Wu's work, we consider that the
methodology proposed in their work allows potential advances in model
selection since it reduces the problem to a feature selection problem.
The last problem is a famous combinatoric problem, with a great myriad
of optimal and suboptimal algorithms, and also a framework for
benchmarking of such algorithms~\cite{REIS2017193}. More than that, in 
instances where chains of the feature selection search space describe U 
shaped curves, it is possible to use algorithms for the U-Curve problem, 
including high parallelizable algorithms~\cite{Estrela2020}.

% To surmount these limitations
% - gather information from many data sources
% - create new search algorithms
% - use Bayesian approach 
For these reasons, in this work, we designed and tested a new method for 
modifying models of signaling pathways, based on the work of Wu, and 
including a possible solution for the third aspect mentioned on the last 
paragraph. Although the first and second aspects have potential of 
improvements, we did not prioritize them and, because of the 
complexity of the third aspect, we decided that including the first and 
second aspect would excessively enlarge the scope of this work.
Therefore, we limit our work to a few known cell signaling pathways, and
we only include simple algorithms for traversing the search space. To
the last aspect, which considers the cost function, as our major 
concern in this work, we intend to use Bayesian approaches to rank 
models~\cite{Vyshemirsky2007}, based on the likelihood of them to 
reproduce the observed data; if we say $M$ is a model with parameter 
space $\Theta$ and $D$ is a set of observations, then we would like to 
estimate 
\begin{equation*}
    p ({\bm D}|M) = \int_{{\bm \theta} \in \Theta} p ({\bm D} | {\bm
    \theta}, M)p({\bm \theta} | M)d{\bm \theta}, 
\end{equation*}
where $p ({\bm D}|{\bm \theta}, M)$ is the likelihood of data ${\bm D}$,
given that the model $M$ with parameters ${\bm \theta}$ are ``correct'',
which is the same as stating that this model and parameters determine
the behavior of the cell; $p({\bm \theta} | M)$ is the prior
probability of ${\bm \theta}$; and finally, $p ({\bm D} | M)$ is the
probability of the data being generated by model $M$. This cost function
has as an advantage the fact that models are not ranked using a single
value for parameters, instead, the cost considers all possibilities of
parameters, integrating over the parameter space. Another advantage of
this cost function is that, since it is based on likelihoods of the
model to reproduce data, overly complex models are automatically
penalized.

\section{Objectives}
In this dissertation, we propose a methodology that allows us 
to solve the problem of identification of cell signaling pathways as a 
feature selection problem, using a Bayesian approach for the cost
function. This cost function should be able to rank models according to
the likelihood of the experimental data being generated by them. Moreover,
we built small examples of model selection of cell signaling pathways, and
run simple searches on their search spaces, using our chosen cost 
function, thus enabling us to get a glance on the
surface that the chosen cost function induces over the search space of
the model selection problem. To achieve these goals, we had to
accomplish the following tasks:

\begin{enumerate}
    \item{{\bf Study state of the art Bayesian algorithms for signaling
        pathway model selection}. We have chosen two methodologies to
        study: the first, a work of Liepe et al.~\cite{Liepe2014}, uses
        an Approximate Bayesian Computation approach for ranking models;
        the second, a work of Xu et al.~\cite{Xura20}, uses an
        estimative of the marginal likelihood $p({\bm D}| M)$ to rank
        models.}
    \item{{\bf Implementation and testing of cost functions}. We proposed
        to compare both the cost functions used by Liepe and Xu. To test
        the cost function used by Liepe, we used the software
        ABC-SMC, and to test the cost function used by Xu, we
        implemented our own software, SigNetMS. After the preparation of
        instances, tested the performance of both software.}
    \item{{\bf Preparation of instances of the identification of cell
        signaling pathway problem.} After comparing both cost functions,
        we constructed instances to test our feature selection
        approach on identification of cell signaling pathway. These
        instances were composed by a base model, a small database of
        candidate reactions, and a set of measurements to which our
        candidate models should fit.}
    \item{{\bf Traversal of the search space.} We performed
        walks and runs of simple search algorithms over the search space
        of candidate models for the instances we created. Those runs
        provided us information regarding the surface that the used cost 
        function induces over the search space. This type of information
        shall be useful for defining new search algorithms for the
        identification of cell signaling pathways.}
\end{enumerate}

\section{Organization}

The remainder of this dissertdissertationn is organized as the following.

\begin{itemize}
    \item{\em Chapter 2 (Fundamental Concepts):} we will give more 
        details about the identification of signaling pathway problem. 
        We will show how to obtain experimental data for signaling 
        pathways and how a chain of chemical reactions can be modeled 
        as a system  of differential equations. We close the chapter 
        presenting briefly state-of-the-art methods of model ranking and
        also the  Metropolis-Hastings algorithm, which is a useful tool
        for methods of model ranking.
    \item{\em Chapter 3 (Model Selection Methods):} we will present two
        methodologies that are the state of the art in model selection.
        Both methods are Bayesian approaches; the first one is based
        on Approximate Bayesian Computation, whereas the second one uses
        Thermodynamic Integration to create an estimate
        of the marginal likelihood of a model.
    \item{\em Chapter 4 (Development of SigNetMS, a Software for Model
        Ranking):} we present the development of SigNetMS, a software
        that allows model selection by creating an estimate of the
        marginal likelihood of a model. We show the main difficulties
        in implementing this software and how we were able to improve
        its performance.
    \item{\em Chapter 5 (Experiments and Results):} we show two 
        different experiments. The first experiment compares two 
        software for model selection: the first, ABC-SysBio, uses ideas 
        of Approximate Bayesian Computation to rank models; the second,
        SigNetMS, is implemented by us and uses ideas of Thermodynamic
        Integration to create estimates of Marginal Likelihood. After
        comparing both, we create an instance of Model Selection problem
        in terms of a Feature Selection problem, and then we proceed to
        use the chosen software as a score function. With the results of
        such experiment, we are able to get a glance of the surface
        induced by the score function over the search space.
    \item{\em Chapter 6 (Conclusion):} we review all the content
        presented in the previous chapters. We also include a list of 
        contributions related to this work. Finally, we include a list
        of possible suggestions of future work related to this
        dissertation.
\end{itemize}


\chapter{Fundamental Concepts}
\label{chap:fundamental_concepts}
%begin-include

% What I want to talk about in this section?
% - Cells signaling pathways are part of the cell communication system,
% and it allows the cell to perceive the conditions of the environment 
% and also to change its behaviour according to the input signal.
% - The signals perceived by a cell can come from cells that very close
% (including itself), as in synapses or it can travel long distances in 
% the organism, as in hormones.
% - When a signal reaches a cell, it can either activate a receptor in
% the cell membrane or diffuse into the cell.
% - Once this happen, the signal or the activated receptor can trigger
% a sequence of chemical interaction, altering the conformation and 
% state of proteins and also changing the concentration of chemical 
% species in the cell. Ultimately, this chain of effects can alter the 
% behaviour of the cell, what is called signal transduction.
% - Studying signaling pathways is important because they help us 
% understand the mechanisms of a cell, which can for example, elucidate 
% treatments for diseases.
\section{Cell Signaling Pathways}
Cell signaling pathways are part of the complex cell communication 
system, and it allows the cell to perceive the conditions of the 
environment in which it is placed and change its behaviour accordingly.
Signaling pathways participate in the regulation of many cell functions,
including development, division and cell death. Bad functioning 
signaling pathways can also be related to diseases, as in many cases of
cancer.

The signal perceived by a cell can come from cells that are close 
(including the same cell that produced the signal), as in synapses, or 
it can travel long distances in the organism, as in hormones. When a 
signal reaches a cell, it can either penetrate the cell or bind to some 
specific receptor in the membrane. Once either of those events happen, 
the signal or the receptor can trigger a sequence of chemical 
interactions that can include change of conformation of proteins, 
activation or inactivation of proteins, and change of concentration of
chemical species in the cell. Ultimately, this chain of chemical 
reactions caused by the signal can alter the behaviour of the cell, what 
is called signal transduction.

Since signaling pathways participate in many of the cell functions, and 
are also related to diseases, it is important to study those structures
in order to get a better understanding of the cell mechanisms and 
diseases. One approach on the study of the cell signaling pathways is to 
measure the concentration change of proteins that participate on the 
pathway of interest.

% - Western blot is a technique that can indicate the amount of a 
% specific protein that is in a mixture of proteins
% - This technique shows the presence of a protein in a mixture by 
% ``blotting'' these molecules into a membrane. 
% - Repeating the procedure in different times define time-course 
% observations of the protein in a biological experiment.
% - With these observations in hand, the researcher is able to construct 
% a measurement that is relevant to describe the biological experiment. 
% For instance, in a signaling network where a protein is closely 
% related to the behaviour of interest of the cell, this protein is a 
% good candidate as a measure of the system.
\section{Measurements of Proteins in Cell Signaling Pathways}
Western blot is a laboratory technique that can indicate the amount of a
specific protein that is present in a mixture. This technique show the 
presence of a protein in a mixture by ``blotting'' a membrane where the 
molecules of interest are located. We can superficially summarize the 
procedure in the following steps: first a mixture containing a sample of 
cells of interest must be created; second, proteins from the mixture 
should be fixed on the blotting membrane; third, an antibody should bind 
to the target protein molecules; and finally, a method for highlighting 
the bound antibody should be applied. An image of the resulting membrane 
can then be analyzed with computer programs to quantify the relative 
concentration (with respect to some other protein, usually a control 
protein that has fairly the same concentration during the whole 
experiment) of the protein of interest.

By repeating this procedure in different times it is possible to create 
time-course observations of proteins throughout the biological 
experiment. With this tool, a researcher can choose a set of relevant 
proteins from a signaling network and gain knowledge about the dynamics 
of such chemical species during the experiment. For instance, in a 
signaling network experiment in which it is desired to understand how
the change of concentrations of a species at the beginning of the 
pathway changes the concentration of some species at the end of the 
cascade, then measurements of both are relevant to understand the 
biological experiment. Figure~\ref{fig:western_blot_example} presents an 
example of time-course Western blot for an experiment where it is 
desirable to understand how extracellular signal-regulated kinase (ERK)
is activated (phosphorylated) as a function of levels of Rat sarcoma
bound to guanosine triphosphate (Ras-GTP).

\begin{figure}[!ht]
\centering
    \includegraphics[width=\textwidth]{fundamental_concepts/western_blot.png}
    \caption{Figure {\bf a} shows time-course measurements of ERK, 
    phosphorylated ERK and hypoxanthine-guanine 
    phosphoribosyltransferase (HPRT). HPRT is a ``loading'' protein, 
    that means that its concentration is fairly the same through the 
    experiment, and therefore it is used as a normalizing factor to 
    total ERK concentration. Figure  {\bf b} shows values of 
    phosphorylated ERK that are obtained after processing figure 
    {\bf a}. Original image of Marcelo S. Reis et al. 
    (2017)~\cite{Reis2017}.}
    \label{fig:western_blot_example}
\end{figure}

These measurements alone do not always provide means for researchers to 
understand a cell signaling pathway experiment. However, if we create a 
computational models for this signaling networks that is able to 
reproduce experimental data, then it is possible to use this model as a 
summary of the signaling network, which can provide to researchers 
evidences of the biological phenomena.


% What do I need to talk about?
% - We can model the concentration changes of chemical species as 
% differential equations, using the mass-action kinetic laws
% - The mass action kinetic law states that in an elementary reaction,
% the rate of a chemical reaction is directly proportional to the 
% product of the reactants concentrations.
% - What is elementary?
% 2.1 Elementary reactions rate equations
% - Types of elementary: first order reaction and second order reaction
% - How do we write up these two? 
% - Chemical notation with constants
% - From there we can write a system of differential equations to 
% describe the signaling pathway.
% - As an example, let's show the equations for a simple enzymatic 
% equation.
% 2.2 Simplifications of reactions
% - More can be done. We can simplify some equations 
% 2.1 Mass conservation simplification
% 2.2
\section{Dynamic Modeling of Cell Signaling Pathways}
One approach onto modeling cell signaling pathways is to model the 
dynamics of the concentrations of chemical species involved. This can be
accomplished when using the law of mass action. This law states that, 
in an elementary reaction, the speed (or rate) of a chemical reaction is 
proportional to the product of the concentration of all reactants. An 
elementary reaction is a reaction in which there is no participation or 
need of an intermediate reaction to describe the first in a molecular 
level. In practice, it is more common to see two types of elementary 
reactions, they are first or second order reactions. 

\subsection{Modeling Elementary Reaction Rates}
A first order reaction is composed of one reactant only. Suppose A is 
the only reactant and B is the only product of a reaction, then we can
write this reaction as:
\begin{equation*}
\ce{
    A -> B
}.
\end{equation*}
The reaction rate of this reaction, according to the law of mass 
action, is 
\begin{equation*}
    k_1[\text{A}],
\end{equation*}
where $k_1$ is some constant and [A] is the concentration of A. 

A second order reaction is composed of two reactants. Suppose C and D 
are both and the only reactants and F is the product of a reaction, then
we can write this reaction as:
\begin{equation*}
\ce{
    C + D -> F
}.
\end{equation*}
The reaction rate of this reaction is:
\begin{equation*}
    k_2\text{[C][D]},
\end{equation*}
where $k_2$ is a constant and [C] and [D] are the concentrations of C 
and D, respectively.

Using these two laws to calculate the speed of reactions, we are able 
to describe how the concentration of chemical species in a system change 
though time using differential equations. To illustrate this and future 
concepts of this section, we are going to consider a minimal system 
composed of a simple enzymatic reaction:
\begin{equation}
\ce{
    E + S <=>[\ce{$k$_f}][\ce{$k$_r}] ES ->[\ce{$k_{cat}$}] E + P
},
\label{eq:simple_enzymatic}
\end{equation}
where E is an enzyme, S is a substrate, ES is the enzyme-substrate
complex, and P is the product.

Each arrow in equation~\ref{eq:simple_enzymatic} represents one 
elementary reaction, and the names over or under arrows represent 
reaction rate constants. All three reactions can be represented by the
equations:

%The first reaction has E and S as reactants and
%ES as a product, and can be written as 
\begin{subequations}
\begin{align}
\ce{
    E + S ->[\ce{$k_f$}] ES 
} \label{eq:es_complex_fwd} \\
\ce{
    ES ->[$k_r$] E + S
} \label{eq:es_complex_rev} \\
\ce{
    ES ->[$k_{cat}$] E + P
} \label{eq:es_pe} 
\end{align}
\end{subequations}
and they have, respectively, reaction rates of:
\begin{equation*}
\begin{aligned}
    & k_f\text{[E][S]} \\
    & k_r\text{[ES]} \\
    & k_{cat}\text{[ES]}.
\end{aligned}
\end{equation*}

{\color{blue} TODO: Maybe I should stop this subsection here and start another
where I explain what comes next as "Dynamic Modeling of a System of Reactions"}

Now, to determine a model of the concentration dynamics of every 
chemical species involved in reaction~\ref{eq:simple_enzymatic}, we will
write a system of ordinary differential equations. To do so, for every 
species we should calculate its concentration change rate based on the 
rate of each reaction that it participates. For instance, the enzyme E is a 
reactant on reaction~\ref{eq:es_complex_fwd} and is also a product on 
reactions~\ref{eq:es_complex_rev} and~\ref{eq:es_pe}, then we consider 
that E changes its concentration over time ($t$) according to the 
differential equation:
\begin{equation}
    \frac{d[\text{E}]}{dt} = -k_f\text{[E][S]} + (k_r + k_{cat}) \text{[ES]}
\end{equation} 
Repeating this procedure for every other species of the enzymatic 
reaction induces the desired system of ordinary differential equations:
\begin{subequations}
    \label{eq:full_system}
    \begin{align}
        \frac{d[\text{E}]}{dt} & =  
            -k_f\text{[E][S]} + (k_r + k_{cat}) \text{[ES]} 
            \label{eq:dEdt} \\
        \frac{d[\text{S}]}{dt}  & = 
            -k_f\text{[E][S]} + k_r\text{[ES]} 
            \label{eq:dSdt} \\
        \frac{d[\text{ES}]}{dt} & =  
            k_f\text{[E][S]} - (k_r + k_{cat}) \text{[ES]} 
            \label{eq:dESdt} \\
        \frac{d[\text{P}]}{dt} & = k_{cat}\text{[ES]} \label{eq:dPdt}.
    \end{align}
\end{subequations}

% Ok, what do I really want to talk about here: Michaelis Menten 
% simplification of enzymtic reactions
% - Before anything, we should use mass conservation to produce the 
% d[ES]/dt equation.
% - Then we should mention the steady-state proposal of Michaelis Menten
%   -> should I use a picture? Maybe I can compare two initial states,
%      one with [S] >> [E] and the other not.
% - Mention that we can suppress a lot of parameters with this 
%   simplification
\subsection{Simplification of Dynamic Models}
The system~\ref{eq:full_system} can be simplified if we
apply properties of enzymatic reactions together with algebraic 
simplifications. We will show then how to derive the quasi-steady-state 
Michaelis-Menten model for enzymatic reactions. With the correct 
assumptions, this model is able to reproduce the behaviour of an 
enzymatic reaction without considering the intermediate enzyme-substrate 
complex.

A basic the principle we need to apply to our system in order to derive
the Michaelis-Menten model is the principle of mass conservation. This 
principle is valid if we assume that the 
reactions~\ref{eq:simple_enzymatic} are isolated, meaning that the 
chemicals on these reactions are not involved in other reactions at the
same time. Applying this principle to the enzyme chemical, produces the
following equation:
\begin{equation*}
    \text{[E$_0$]} = \text{[E] + [ES]}.
    \label{eq:E_conservation}
\end{equation*}
If we apply this equation to the derivative of the concentration of ES,
we will get the following equation:
\begin{equation}
    \frac{d[\text{ES}]}{dt} =  
        k_f(\text{[E$_0$]} - \text{[ES]})\text{[S]} 
        - (k_r + k_{cat}) \text{[ES]}. 
        \label{eq:dESdt_2}
\end{equation}

One more assumption is necessary to derive the simplification. This 
assumption states that the concentration of substrate-enzyme complex
does not change over time, i.e. $\frac{d[\text{ES}]}{dt} = 0$, and it 
was first proposed in 1925 by Briggs and Haldane~\cite{Briggs1925}. 
Generally, this assumption is applicable whenever [S] $\gg$ [E]. 
{\color{blue} TODO: and we can apply it on this work because...}
Applying this assumption together with the mass conservation assumption 
on the equation~\ref{eq:dESdt_2}, we get:
\begin{equation*}  
    \begin{aligned}
        \text{[ES]} (k_r + k_{cat}) &= 
            k_f(\text{[E$_0$]} - \text{[ES]})\text{[S]}, \\
        \text{[ES]} &= \frac{\text{[E]}_0\text{[S]}}{K_m + \text{[S]}} 
    \end{aligned}
\end{equation*}
in which $K_m = \frac{k_{cat} + k_r}{k_f}$ is known as Michaelis 
constant. Considering this, we can rewrite the rate of [P] as:
\begin{equation*}
    \frac{d\text{[P]}}{dt} = k_{cat}\frac{\text{[E]}_0\text{[S]}}
        {K_m + \text{[S]}}
\end{equation*}
And finally, if we apply mass conservation to the substrate, we will get
the following equation:
\begin{equation*}
    \text{[S$_0$]} = \text{[S] + [ES] + [P]},
\end{equation*}
the, we can differentiate this equation on $t$ and use the 
quasi-steady-state assumption ($\frac{d\text{[ES]}}{dt} = 0$) to obtain:
\begin{equation*}
    \frac{d\text{[S]}}{dt} = - \frac{d\text{[P]}}{dt}
\end{equation*}

Therefore, using the Michaelis-Menten model, we are able to simplify the
system~\ref{eq:full_system} that had four equations and three 
parameters to a new model that has only two equation and two parameters 
($k_{cat}$ and $K_m$). Figure~\ref{fig:michaelis_menten} shows a 
comparison between the complete and Michaelis-Menten models of enzymatic
reactions.

\begin{figure}[H]
  \centering 
  \begin{tabular}{c c}
    \subfigure[] {\scalebox{1}{
    \includegraphics[trim={0 0 0 1.4cm}, clip=true, width=.45\textwidth]{fundamental_concepts/simplifications/full_system.png}}
     \label{fig:enzymatic_full}}
     &
    \subfigure[] {\scalebox{1}{
    \includegraphics[trim={0 0 0 1.4cm}, clip=true, width=.45\textwidth]{fundamental_concepts/simplifications/mm_system.png}}
    \label{fig:enzymatic_mm}}
  \end{tabular}
    \caption{An example of the dynamics produced by two models of 
        enzymatic reactions. The figure~\ref{fig:enzymatic_full} 
        presents the dynamics of the model~\ref{eq:full_system} and 
        figure~\ref{fig:enzymatic_mm} presents the dynamics of the 
        Michaelis-Menten simplification to the same model. For this 
        simulations, it is necessary to define initial concentrations of
        the chemical species involved, and it is used: 
        10 molecules$/(\mu m)^3$ for the enzyme (E); 
        100~molecules$/(\mu m)^3$ is used for the substrate (S); and 
        0~molecules$/(\mu m)^3$ is used for the other species. In 
        addiction to this, it is also necessary to define model 
        parameter values, and it is used: 
        0.06~$(\mu m)^3$(molecules$*s)^{-1}$ for $k_f$; 
        0.1~$(s^{-1})$ for $k_r$; 0.2~($s^{-1}$) for $k_{cat}$; and,
        following the Michalis-Menten model, 5 molecules$/(\mu m)^3$
        is used for $K_m$.}
  \label{fig:michaelis_menten} 
\end{figure}


\section{Identification of Cell Signaling Pathways}
\section{State of the Art in Model Selection}
%\section{Bayesian Inference}
%\subsection{Girolami (bioinformatics)}
%\subsection{Kolch}
%\subsection{ABC-Sysbio}


\chapter{Model Selection Methods}
\label{chap:model_selection}
%begin-include
{\color{blue} Simple description of the content of this chapter}.

\section{Model ranking using Marginal Likelihood}
% Simple explanation of this model ranking
% Describe the likelihood function 
% However, this is very hard to calculate
% It is possible to apply an Importance Sampling Estimator {cite 
% Newton and Raftery}. However, it was showed on {cite girolami again}
% that these methods do not perform well.
% Then it was proposed to use Thermodynamic Integration. Make it clear 
% that this allows us to create some estimators.
% Subsection - Thermodynamic Integration
% -> explain how to derive it
%   -> name what is the power posterior
% -> include here: how to estimate it?
% Subsection - Estimation of the Marginal Likelihood 
% -> We need to find samples of the posteriors
% -> Explain that we use the methodology of Xuan
% -> We use three Metropolis-Hastings
% -> A first burn-in step with jumps independent for each single 
% parameter. Adaptive Metropolis Hastings
% -> A second using a Variance-Covariance Matrix
% -> A third using populational MCMC
The marginal likelihood of an experiment measurement $D$ being 
reproduced by a model $M$, $p (D | M)$, can be used as a model ranking 
metric as it determines which model makes the experimental observations 
more likely to happen. Before defining how to calculate the  marginal 
the likelihood, we must define what the likelihood function is. To 
calculate the likelihood $p (D | M, \theta)$, we must understand that 
conditioning the observation to the model and parameters means that in 
the probability space from which $D$ is taken, the model $M$ is the 
``real'' model and it has the parameter values of $\theta$; i.e. the 
model $M$ with parameters $\theta$ controls the behaviour of the system 
from which $D$ was observed. Then, assuming that the observations have a 
Gaussian error, and that they are taken in a time series of $m$ time 
steps, we can define the likelihood as:
\begin{equation}
    p (D | M, \theta) = p_{\mathcal{N}_{\left(\vec{0}, \Sigma\right)}}
        (\phi (M,\theta) - D),
\label{eq:likelihood_multivariate}
\end{equation}
where $\phi (M, \theta) \in \fieldR^m$ is the experimental measurement 
on the simulation generated by the model $M$ with parameters $\theta$,
and \smash{$p_{\mathcal{N}_{(\vec{0}, \Sigma)}} (\cdot)$} is the 
probability density function of a Multivariate Normal variable with mean 
$\vec{0}$ and covariance matrix $\Sigma$. As a matter of fact, as it is
done in the work of Xu et al., we can consider that the observation 
error is independent for each time step~\cite{Xura20}, therefore we can 
simplify~\ref{eq:likelihood_multivariate} to:
\begin{equation}
    p (D | M, \theta) = \prod_{i = 1}^m p_{\mathcal{N}_{\left(0, 
        \sigma^2\right)}} (\phi_i (M,\theta) - D_i).
\label{eq:likelihood}
\end{equation}
The $\sigma^2$ used in equation~\ref{eq:likelihood} is also a parameter
of the model, which means that, for some $k$, $\theta_k = \sigma^2$.

Now that we defined the likelihood function, we can write the marginal 
likelihood as:
\begin{equation}
    p (D | M) = \int_{\Theta} p (D | M, \theta) p (\theta | M)d\theta.
\label{eq:marginal_likelihood}
\end{equation}
However, calculating this integral analytically is only possible in 
very special cases and, usually, it would depend on knowing models for 
the distributions associated to these probability functions, which is 
generally not possible in our case.

Even though this integral is very hard to be calculated, there are 
methods that allow us to estimate its value. A straight forward method 
to estimate this integral value is the Importance Sampling 
Estimator~\cite{Newton1993}. This method uses the Monte Carlo integral 
estimation method that can estimate integrals of the form 
$\int g(\lambda) p(\lambda)d\lambda$ using the estimator:
\begin{equation}
    \hat{I} = \sum_{i = 1}^m w_i g(\lambda_i) / \sum_{i = 1}^m w_i,
\label{eq:importance_sampling_estimator}
\end{equation}
where $w_i = p (\lambda) / p^* (\lambda)$, and $p^*(\cdot)$ is known as 
the importance sampling function. If we set $\lambda = \theta | M$ and
use the prior ($p(\theta | M)$) or the posterior ($p(\theta | M, D)$) as 
importance sampling functions, then we would get respectively the 
estimators:
\begin{equation}
\begin{aligned}
    \frac{1}{m} \sum_{i = 1}^m p(D|M, \theta^{(i)}) &&& 
        (\text{with } \theta^{(i)} \sim p(\theta|M)); \\
    \left(\frac{1}{m} \sum_{i = 1}^m p(D|M, \theta^{(i)})^{-1} \right)^{-1} &&&
        (\text{with } \theta^{(i)} \sim p(\theta|M)).
\end{aligned}
\end{equation}
However, as showed by Vyshemirsky et al. (2007), these estimators might
produce very large variances and may not perform well for biochemical
model selection applications. For that reason we decide to use a 
Annealing-Melting method as it is proposed in the same 
work~\cite{Vyshemirsky2007}.

% ABC-SysBio
% Experiments comparing both approaches


\chapter{Development of SigNetMS, a Software for Model Ranking}
\label{chap:development_signetms}
%begin-include
In this chapter we present the development process and implementation
details of the SigNetMS software, which stands for {\bf Sig}naling 
{\bf Net}work {\bf M}odel {\bf S}election. This software is capable of
producing an estimative of the marginal likelihood of a model given
experimental data, $p({\bm D} | M)$.

% What are we going to talk about in this chapter?
% - first, we have to talk about which methodology this software uses
%   -> make sure you citate BioBayes and how its cumbersome to use it
% to create a model ranking. we should state that we are using marginal
% likelihood estimates here.
%   -> make sure we specify the input and output of this software
%   -> include program arguments
% - then, we can talk about the implementation and optimizations
%   -> how did we implement the sampling? what was the used proposal
%   distribution?
%   -> fast integration of system of differential equations
%   -> parallel sampling
%   -> running the algorithm on a cluster

\section{The SigNetMS Software}
% General Aspects of the software

% The Bayesian methodology
% - SigNeMS creates an estimative of the marginal likelihood of a model
% given a set of experimental data.
% - To create this estimative, SigNetMS uses ideas of Thermodynamic
%   Integration.
% - There is a software that can carry similar simulations, called
%   BioBayes, however we found its use cumbersome, mainly because:
%   it is a GUI software, which does not fit our type of applications,
%   that should be ran in a server for several hours. Moreover, we
%   intended to link the marginal likelihood output to other programs,
%   for model selection.

% Details on how the program creates the system of ODEs
% Details on how the proram estimates the likelihood

SigNetMS is a Python program that can be used as a tool for model
selection. The source code is available on 
Github\footnote{https://github.com/gustavoem/SigNetMS} and it is open
source, under the GNU General Public License.

This program expects as the input: a signaling pathway model,
represented by a Systems Biology Markup Language
(SBML)~\cite{hucka2003systems} file, with the definition of reactions,
kinetic laws and initial concentrations of chemical species; an
Extensible Markup Language (XML) file with experimental data, including 
time series measurements of the  biological phenomena of interest;
another XML file with definitions of prior distributions of reaction
rate constants; and, finally, a set of parameter values that determine
the sampling process of model parameters. There are also optional
parameters on SigNetMS, used to control random number generator seeds,
number of execution threads, and verbose runs.

The output of the program is composed by an estimative of the marginal
likelihood of the model, given experimental data, $p({\bm D} | M)$ and a
list of parameter values $({\bm \theta}^1, {\bm \theta}^2, \ldots {\bm
\theta}^l)$ that represent a sample of the distribution $p({\bm \theta}
| M, {\bm D})$. If one simulates the model $M$ with parameter values
from the sample, we expect that, the higher the marginal likelihood, the
closer the simulation is to the experimental data.

To calculate this estimative, we use the ideas of Thermodynamic
Integration we presented on section~\ref{sec:thermodynamic_integration}.
Before implementing SigNetMS we also considered using another software,
called BioBayes~\cite{Vyshemirsky2008}, that has a very similar approach
to produce an estimative of the marignal likelihood. However, we did not
proceed to use this software because there was only a graphical user
interface version of the software, which made the process of running
instances and collecting results cumbersome. We have also tried
contacting the authors to obtain the source code, however they could not
provide us that.

\section{Creating an estimative of the marginal likelihood}
% Since we are using a Thermodynamic Integration  approach, we need to 
% define a likelihood function, create samples of power posterior 
% distributions and, finally define an approximation to the marginal
% likelihood.

% The likelihood function...
% The samples of power posteriors are...
% The approximation of the marginal likelihood are...
% The whole process, has the following order:
% -> all inputs are read;
%   -> The sbml model is read and transformed into a System of ODE
%       -> this model can determine a simulation, given a list of
%       parameter values and a measurement unit;
%   -> Parameter priors, experimental data are read and saved
% -> sampling of posterior starts with naive sampling
% -> then we proceed to covariance shaped burn-in
% -> then we go to populational monte carlo markov chain
% -> then we use the approximation to generate the margginal likelihood

To create estimates of the marginal likelihood using the Thermodynamic
Integration approach, we need to define a likelihood function $p({\bm
D}| M, {\bm \theta})$ and also samples power posterior distributions
$p_\beta({\bm \theta})$ for a sequence of values of $\beta$. On
SigNetMS, we use the trapezoidal rule to approximate the marginal
likelihood, given by the
equation~\ref{eq:marginal_likelihood_trapezoidal_approximation}. 
Inspired by the work of Friel et al., we discretize the
interval $[0, 1]$ of power posteriors using the $\beta_i =
\left(\frac{i - 1}{T - 1}\right)^c$, where $T = 20$, $c = 4$ and $i \in
\{1, 2, \ldots, T\}$.

% TODO: say here that we used pipipi popopo beta values and that with
% samples of those power posteriors, we estimated the marginal
% likelihood as...

\subsection{Implementing the likelilhood function}
%-> The likelihood function is...
%-> To implement this, we must simulate the model with parameter values
%of theta.
%-> these simulations involves the numerical integration of a system of 
% ordinary differential equaitons
%-> current methods of numerical integrations are iterative and can be
% very consty depending on the size of the numerical system and also on
% the type of the system, stiff or not
The likelihood function we used is the same as we defined in
equation~\ref{eq:likelihood_multivariate}:
\begin{equation*}
    p ({\bm D} | M,{\bm \theta}) = 
        p_{\mathcal{N}_{\left(\vec{0}, \Sigma\right)}}
        (\phi (M, {\bm\theta}) - {\bm D}),
\end{equation*}
where ${\bm D}$ is the experimental measurement, $M$ is the model,
${\bm \theta}$ is a set of parameter values, $\Sigma$ is the variance of
the error of experimental observations, and, finally, $\phi$ is a 
function that calculates an approximation of the results of the
experiment that produced ${\bm D}$, applied to the simulated environment
of model $M$ with parameter values ${\bm \theta}$. This simulation of
the model is created by deriving a system of ordinary differential
equations, and then numerically integrating this system, over the time
steps defined by the experimental measurements. To reproduce the same
experiment that generated ${\bm D}$, SigNetMS expects that the XML file 
containing experimental data also contains a mathematical representation
of which quantity was measured, in terms of concentrations of chemical
species of the system.

% Numerical integration of systems are solved using iterative methods
% stiff non stiff
% we used  third-party software to solve this problem
The numerical integration of the system is alone a hard problem, and
therefore, we used third-party software to produce such integrations.
The most popular software available for this problem conduce
iterative algorithms that, step by step, aproximate the state of the
system for a time interval. It is important to know that some instances
of the problem can be stiff, meaning that they may make the integration
algorithm be unstable, since it may need consecutive iterations in 
really small steps. Since we did not have time to go in details of when 
such cases occur, we chose third-party softwares that can adapt the used
algorithm according to the stiffnes of the instance.

After the implementation of the function $\phi$, most of the work to
implement the likelihood function is done. The reamining work is to
calculate the value of the probability density function
$p_{\mathcal{N}_{\left(\vec{0}, \Sigma\right)}}$ and that can easily be
accomplished using statistical packages such as
SciPy~\cite{2020SciPy-NMeth}.

\subsection{Sampling parameters from power-posteriors}
% after implementing the likelihood function, we can create the samples
% of the power posteriors;
% -> we used the same methodology we presented before, this methodology
%  is based on using variations of the metropolis-hastings algorithm,

After implementing the likelihood function, we can move to the creation
of samples of power posteriors. The methodology we used to generate such
samples is identical to the one we presented on
section~\ref{sec:estimation_of_marginal_likelihood}. And for this
reason, we divided the sampling process in three phases: naive burn-in,
adaptive burn-in and Populational MCMC. All of these process are types
of a Metropolis-Hastings procedure. The number of iterations of each
phase is determined by SigNetMS's arguments, and each phase has a
different scheme to determine the proposal distribution.

On the first phase, we start the sample of every chain (for each value
of $\beta$) with a random draw from the prior distribution of
parameters. Before the first iteration, we also create an estimative of
the variance of the logarithm of each model parameter, independently.
These estimates are used to build the first covariance matrix of the
jumping distribution; we use a diagonal matrix where the diagonal
elements are set as the estimated variance of the logarithm scaled
sample of the associated parameter. This matrix is rescaled according to 
the acceptance rate, as described in
section~\ref{sec:estimation_of_marginal_likelihood} after a number of
iterations that is defined in one of arguments of SigNetMS. For each
iteration, we determine that the jumping distribution is a multivariate
lognormal $({\bm \mu}, \Sigma)$ distribution, with covariance matrix as
explained before, and with ${\bm \mu} = \log_e({\bm \theta}^t)$, where
${\bm \theta}^t$ is the current sample point. That means we take a
sample ${\bm X}$ of the multivariate normal $\mathcal{N}({\bm \mu},
\Sigma)$ and then we set our sampled value  as ${\bm Y} = \exp({\bm
X})$, which is a standard procedure to produce samples of lognormal
distributions.

On the second phase, the posterior shaped burn-in phase, we also use a 
lognormal as the proposal distribution, however the covariance matrix is 
not diagonal. Half of the sample produced in the first phase is
discarded, and for each step, we calculate the covariance of the
log-scaled current sample, producing the matrix used as the covariance
matrix for the proposal distribution. Similarly to the first phase, the
proposal distribution also uses ${\bm \mu} = \log_e({\bm \theta}^t)$. It
is important to remember that up to the end of this sampling phase, each
power posterior sample is produced independently.

The third and last phase, we perform a Populational Monte Carlo Markov
Chain. In this procedure, we iterate each chain of power posterior
samples using the same algorithm of the second phase (except we do not
update the covariance matrix anymore), followed by an exchange the last
sampled points on two random selected power posteriors. At the end of
this phase, we discard parameters sampled on previous phases and we set
the actual sample as all the parameters sampled in this phase.


\chapter{Experiments and Results}
\label{chap:experiments}
%begin-include

In this chapter we present experiments of model selection of cell
signaling pathways, and we divide the contents of this chapter in two 
sections. The first section presents experiments to compare two model 
selection software: ABC-SMC and SigNetMS, both of them presented on 
Chapter~\ref{chap:model_selection}. To make this comparison, we
use an instance of model selection with only four candidate models,
evaluating the score of each one of them. The following section presents
experiments of another instance of model selection, however, this time
formulated as a feature selection problem. We will then analyze the
search produced by the feature selection algorithm and how it relates to
a ``correct'' model, which was used to create the experimental
measurements.

\section{Choosing a software for model selection}
% To choose a software for model selection we did the same experiment as
% girolami
%
% A simple instance of the model selection problem
% -> the correct model is...
% -> then we create other three models
%   - a simplification
%   - an overly complex model
%   - and an incorrect model
% -> the expected result for this experiment is that the correct model
%  has a higher (possibly be the first), and that the incorrect model
%  should be considered the worst. It is also important to see how the
%  software compares models with similar dynamics and with different
%  levels of complexity.
%  
% Results produced by SigNetMS and ABC-SysBio
% -> we proceeded to run both softwares 
% -> compare the ranking of both software
% -> show how the curve fits on both software
% -> show that there is some parameter value convergence on SigNetMS

To compare results of SigNetMS and ABC-SysBio, we performed a simple 
model selection experiment. This experiment, originally performed on the 
work of Vyshemirsky and Girolami~\cite{Vyshemirsky2007}, consists in 
creating artificial experimental data from a model of cell signaling 
pathway, and then selecting between four different models, including the
correct one. Using SigNetMS and ABC-SysBio we should be able to create a
ranking of the four models, in which we expect to see as the best, the
model we used to create the experimental data. More than that, we should
analyze the produced results to check if simpler models are preferred 
over complex models; we should also check if the simulations produced by
the models, with the estimated sample of the posterior distribution of
parameters, approximates experimental data.

\subsection{A simple instance of the model selection problem}
\begin{figure}[h]
\begin{center}
    \includegraphics[width=.75\textwidth]{experiments/diagrams/bioinformatics_model1.pdf}
    \caption{A diagram that represents the correct model our simple
model selection experiment. This model represents a common motif, and
has S and Rpp as input and output, respectively. This model contains
five reactions: the decay of S to dS with $k_1$ as reaction rate
constant; the reversible reaction $\ce{S + R <=>[$k_1$][$k_2$] RS}$; the
first order reaction $\ce{RS ->[$k_4$] Rpp}$; and the Michaelis-Menten
reaction $\ce{A ->[$V, K_m$] B}$.}
    \label{fig:experiments:girolami_model1}
    \end{center}
\end{figure}

We start describing our model selection problem with the correct model,
which is a signalling pathway composed by five reactions and five 
chemical species. Figure~\ref{fig:experiments:girolami_model1} shows a 
diagram of this model. This model represents a common motif, and has
as the input signal the chemical species S, and as the output the
chemical species Rpp. The experimental measurement is the concentration
of the output chemical species, which we donote as [Rpp].

In this experiment, for the sake of simplicity, we neglect the units of 
both reaction rates constants and initial concentrations. The initial 
concentrations used are:  S $= 1$, R $= 1$, dS $= 0$, RS $= 0$, 
R$_{pp} = 0$. To create the experimental data, the reaction rate 
constants we used have the values: 
$k_1 = 0.07$, $k_2 = 0.6$, $k_3 = 0.05$, $k_4 = 0.3$, $V = 0.017$, and
$K_m = 0.3$. It is important to remember that we discard reaction rate
constant values during model selection; initial concentrations, however,
are still provided during this phase. To generate experimental data, we
simulate the dynamics of this model, using these parameter values, on
the time steps of: 2, 10, 20, 40, 60 and 100 seconds. Three simulations
are created, and to each one of them we add, for each time measurement,
a Gaussian error with mean $0$ and standard deviation $0.01$. A
representation of the three experiment repetitions are showed on 
figure~\ref{fig:experiments:girolami_simulations}.

% insert simulated data here
\begin{figure}
\begin{center}
    \includegraphics[width=.75\textwidth]{experiments/simulations/girolami_experimental_data.pdf}
    \caption{The dynamics produced by the correct model, with
pre-defined reaction rate constants plus a small Gaussian error, for
each time point. The measurement taken from the model is the
concentration of the Rpp species, which we denote as [Rpp]. We
linearly interpolate the experimental measure points to produce a
continuous dynamics from 2s to 100s.}
    \label{fig:experiments:girolami_simulations}
    \end{center}
\end{figure}

To assess the ranking produced by each of the model selection software,
we compare the first model with three other models, all of them built as
modifications of the ``correct'' model: a simplified model; an overly 
simplified model, which should not be able to generate the observed 
dynamics; and, finally, a generalization (more complex) model.
Figure~\ref{fig:experiments:girolami_other_models} shows diagrams that
represent the three alternative models.

\begin{figure}[h]
    \centering
    \begin{tabular}{c c}
    \subfigure[simplified model]{
    \includegraphics[clip=true,width=.45\linewidth]{experiments/diagrams/bioinformatics_model2.pdf}
    \label{fig:girolami_model2}}
    &
    \subfigure[overly simplified model]{
    \includegraphics[clip=true,width=.45\linewidth]{experiments/diagrams/bioinformatics_model3.pdf}
    \label{fig:girolami_model3}} 
    \\
\multicolumn{2}{c}{    
    \subfigure[generalization model]{
    \includegraphics[clip=true,width=.45\linewidth]{experiments/diagrams/bioinformatics_model4.pdf}
    \label{fig:girolami_model4}}
} 
    \end{tabular}
    \caption{The diagrams of three other candidate models, based on the 
correct model that was presented before on 
Figure~\ref{fig:experiments:girolami_model1}. The 
model~\ref{fig:girolami_model2} is a simplification where we neglect the
chemical species RS, and we use the Michaelis-Menten to represent the
reaction $\ce{R -> Rpp}$ with S working as a catalyst.
Model~\ref{fig:girolami_model3} is the over simplified model, as it
neglects the decay of S; we do not expect this model to reproduce
experimental data, since the constant concentration level of S tend to
continuously produce Rpp, a species that, after 20 seconds, has a
monotonic decreasing concentration. Finally,
Model~\ref{fig:girolami_model4} is a generalization of the correct
model, as it generalizes the reaction $\ce{Rpp -> R}$, as instead of 
using the Michaelis-Menten kinetics, we use the enzymatic reaction 
$\ce{Rpp + PhA <=> RppPhA -> R + PhA}$; even though we expect this model
to be able to reproduce observed dynamics, we also expect that the
complexity of this model gets penalized.
}
    \label{fig:experiments:girolami_other_models}
\end{figure}

% -> the expected result for this experiment is that the correct model
%  has a higher (possibly be the first), and that the incorrect model
%  should be considered the worst. It is also important to see how the
%  software compares models with similar dynamics and with different
%  levels of complexity.

Before talking about results produced by different software, we should
note that this choice of candidate models are made so we can analyze
more than the ability of the software to correctly rank the correct
model as the best model. First, consider that we introduced a spurious
model, represented on figure~\ref{fig:girolami_model3} which neglects a
crucial reaction, making it impossible to reproduce the experimental
data; we expect this model to be ranked last between all models. Then,
there are two options to the correct model, one a simplification, and
the other a generalization. For these models, we expect that the
experimental dynamics are possible, however, we should be observant of
how they are ranked according to their complexity. That is important
because, one of the goals on using a Bayesian approach for model 
selection is that these approaches tend to automatically penalize overly
complex models.

% Comparing results of ABC-SMC and SigNetMS
\subsection{Solving a simple model selection instance using ABC-SysBio
and SigNetMS}
% What is the experiment
% What data the experiment produces;
% What are algorithm parameters we used;
% 1 - marginal likelihood of each model
% 2 - a sample of the posterior

% the input and output
After defining the candidate models and producing the artificial
experimental data, we proceeded to perform the experiment of model
selection. The instance information provided to SigNetMS and ABC-SysBio 
is the same: a model, with predefined initial concentrations of chemical 
species; a set of experiments, with the same time steps, and with
measurements of the concentration of Rpp; and a file containing prior
distributions for each one of the model parameters. It is important to
remember that the output produced by each software is different.
SigNetMS produces an estimative of $p({\bm D} | M)$ and also a sample of
the posterior distribution of parameters $p({\bm \theta} | M, {\bm D})$
which is, in fact, composed by samples of all power posterior
distributions $p_{\beta}({\bm \theta})$, as we described
on~\ref{sec:creating_an_estimative_of_the_marginal_likelihood}.
ABC-SysBio, on the other hand, produces estimatives of $p({\bm
\theta}, M | {\bm D})$ that tend to be closer to this target 
distribution on each iteration. Note that in this experiment, we need to
run SigNetMS for every model, while on ABC-SysBio we only need to run 
the software once for all four candidate models.

The prior distribution of parameters are the same as used by Vyshemirsky
and Girolami~\cite{Vyshemirsky2007}. All model parameters priors are
Gamma(1, 3), where the first and second arguments are shape and scale,
respectively. Gamma and Lognormal distributions are often used as prior 
for parameters because they have a zero probability density for negative
values.
% what are algorithm parameter values used

For ABC-SysBio, we decided to use its feature of automatically choosing
the schedule of threshold values, which is based on the acceptance of
produced individuals on each iteration. For SigNetMS, we used the
following parameters values: 15000 iterations of the naive burn-in, and
5000 iterations of the posterior shaped burn-in, with 1000 iterations
between covariation matrix rescales, and 3000 iterations of the
Populational MCMC. We used an empiric approach to determine these
parameter values, observing similar results when the number of
iterations are greater than these.

\subsubsection{The ranking produced by ABC-SysBio and SigNetMS}
The ABC-SysBio run created 26 populations of parameter values, each of 
them with 100 individual parameters values. At the last iteration, the 
algorithm stopped with $\epsilon = 1$ and the following estimates: 
\begin{itemize}
    \item{$\hat{p} (M = \text{Correct Model} | {\bm D}, \epsilon = 1) =
        0.005$;}
    \item{$\hat{p} (M = \text{Simplified Model} | {\bm D}, \epsilon = 1)
        = 0.014$;} 
    \item{$\hat{p} (M = \text{Incorrect Model} | {\bm D}, \epsilon = 1)
        = 0.976$;}
    \item{$\hat{p} (M = \text{Generalization Model} | {\bm D}, \epsilon
        = 1) = 0.003$.}
\end{itemize}
These estimates induce the ranking: Incorrect Model $\prec$ Simplified 
Model $\prec$ Correct Model $\prec$ Generalization Model.

After running the SigNetMS software four times, one for each model, we
were able to get the following estimates:
\begin{itemize}
    \item{$\log \hat{p}({\bm D} | M = \text{Correct Model}) = 26$}
    \item{$\log \hat{p}({\bm D} | M = \text{Simplified Model}) = 21$}
    \item{$\log \hat{p}({\bm D} | M = \text{Incorrect Model}) = -1$}
    \item{$\log \hat{p}({\bm D} | M = \text{Generalization Model}) =
        19$}
\end{itemize}
These estimates induce the ranking: Correct Model $\prec$ Simplified
Model $\prec$ Generalization Model $\prec$ Incorrect Model.

% subsubsection Comparing the produced ranking
\subsubsection{Comparing the ranking produced by ABC-SysBio and SigNetMS}
Before comparing the model ranking produced by ABC-SysBio and SigNetMS,
we should state that the ranking achieved by Vyshemirsky and 
Girolami~\cite{Vyshemirsky2007}, on the original work that introduced
this instance, is: Correct Model $\prec$ Generalization Model $\prec$
Simplified Model $\prec$ Incorrect Model. On this work, a methodology
similar to SigNetMS was used.

On ABC-SysBio results, we see that the Correct Model was not ranked
first, and, surprisingly, the Incorrect Model was ranked first. More
than that, when the algorithm stopped, other candidate models were
considered with low probability of being the ``true'' model, and
therefore we cannot strongly state a ranking between the other three
candidates.

On SigNetMS results, we see that the Correct Model was ranked first and 
the Incorrect Model is ranked last as expected. For these two models,
SigNetMS results are equal to the results of Girolami and Vyshemirsky,
and for the other two models, the ranking is the opposite. On SigNetMS,
we ranked the Simplified Model as better than the Generalization Model.
It is important to note here that, in fact, the Generalization Model,
which is more complex, was actually ranked worse than the Correct Model;
that is an evidence that this approach does penalize the complexity of
models.

% subsubsection Analyzing the distribution of posteriors
\subsubsection{Analyzing the posterior distributions produced by
ABC-SysBio and SignetMS}
If we consider only the ranking produced, there are indications that 
SigNetMS is a better choice for our application. However, we should also
take into account other output information produced by both software, 
relative to the distribution of model parameters. ABC-SysBio algorithm 
produces in every iteration a population of parameters that, when 
applied to an specific model, creates a simulation that is at most 
epsilon distant to the experimental measurements, with decreasing 
epsilon as the iteration number grows. SigNetMS, on the other hand, 
produces samples of forty power posterior distributions 
$p_\beta({\bm \theta})$, and although there is no threshold like there 
is on ABC-SysBio, we expect that the closer the value of $\beta$ is to 
$1$, the closer should be the produced simulation to the experimental 
measurements; this is explained by the fact that the power posterior 
distributions $p_0({\bm \theta})$ and $p_1({\bm \theta})$ are, 
respectively, the prior and posterior distribution of parameters.

% One way we could analyze data is looking at the produced simulations
A possible approach to analyze the produced parameters is to simulate
models with those parameter and create simulations to be compared with
the experimental measurements. For ABC-SysBio, in a population of 100
parameters, including the model indicator as one of the parameters, we
are able to simulate and visualize the generated experimental 
measurements for all individuals; on SigNetMS, on the other hand, the 
number of parameter values produced is much greater, than a randomly 
chosen subset of parameters should be enough. With such experiment, we 
are then able to identify what dynamics were created on the candidate 
models according to the estimated posterior distribution of parameters. 

On  figure~\ref{fig:girolami_allmodel_abc} we present the dynamics of 
sampled parameters of the last iterations of the ABC-SysBio run. We can
see on this figure that ABC-SysBio could not produce a set of parameter
values that allows the model to represent the dynamics observed on the
experiment. More than that, we can see that the incorrect model had the
best fit, and the dynamics produced by the sampled parameters induces a
nearly stationary dynamics of $[Rpp]$, with intermediary values of
concentration. With these results, we can understand that the ranking
produced by ABC-SysBio is incorrect because the software could not find
suitable parameter values that allow models to approximate the dynamics
observed on experiments.

\begin{figure}[ht]
    \centering
    \begin{tabular}{c c}
    \subfigure[correct model]{
    \includegraphics[clip=true,width=.45\linewidth]{experiments/abc_vs_snm/all_model/abc/msimulations_model1_25.pdf}
    \label{fig:girolami_model1_abc}}
    &
    \subfigure[simplified model]{
    \includegraphics[clip=true,width=.45\linewidth]{experiments/abc_vs_snm/all_model/abc/msimulations_model2_25.pdf}
    \label{fig:girolami_model2_abc}} 
    \\
    \subfigure[incorrect model]{
    \includegraphics[clip=true,width=.45\linewidth]{experiments/abc_vs_snm/all_model/abc/msimulations_model3_25.pdf}
    \label{fig:girolami_model3_abc}}
&
    \subfigure[generalization model]{
    \includegraphics[clip=true,width=.45\linewidth]{experiments/abc_vs_snm/all_model/abc/msimulations_model4_25.pdf}
    \label{fig:girolami_model4_abc}}
    \end{tabular}
    \caption{The simulated dynamics of the four candidate models, using
    parameters generated on the ABC-SysBio software. There are 100
    individuals generated on each iteration of the algorithm, and each
    individual consists of a list of parameter values and a model
    indicator. Because of this, the number of produced simulations is 
    not equal between models, in fact, the better the fit of a model,
    the higher the number of individuals representing such model, and
    therefore, the higher the number of simulations shown. Each red line
    represent an individual simulation, and stronger red lines represent
    overlapping simulations. Lines with blue, yellow and green color
    represent experimental observations.}
    \label{fig:girolami_allmodel_abc}
\end{figure}

On figure~\ref{fig:girolami_allmodel_snm} we present the dynamics of a 
subset of parameters of the posterior distribution (or power posterior 
of $\beta = 1$), for all four candidate models. We can see on this
figure that SigNetMS could not find parameter values that allow the
incorrect model to reproduce experimental observations, which is
expected, and that the three other models candidate models could
closely reproduce the experimental observations. It is also interesting
to observe the dynamics produced by sampled parameters for other values
of $\beta$, which is shown on figure~\ref{fig:girolami_model1_progression_snm},
for the correct model only. Remember that from $\beta = 0$ to 
$\beta = 1$, a sequence of power posterior distributions is constructed 
by SigNetMS, bridging the prior and posterior distributions.

\begin{figure}[ht]
    \centering
    \begin{tabular}{c c}
    \subfigure[correct model]{
    \includegraphics[clip=true,width=.45\linewidth]{experiments/abc_vs_snm/all_model/snm/msimulations_model1_39.pdf}
    \label{fig:girolami_model1_snm}}
    &
    \subfigure[simplified model]{
    \includegraphics[clip=true,width=.45\linewidth]{experiments/abc_vs_snm/all_model/snm/msimulations_model2_39.pdf}
    \label{fig:girolami_model2_snm}} 
    \\
    \subfigure[incorrect model]{
    \includegraphics[clip=true,width=.45\linewidth]{experiments/abc_vs_snm/all_model/snm/msimulations_model3_39.pdf}
    \label{fig:girolami_model3_snm}}
&
    \subfigure[generalization model]{
    \includegraphics[clip=true,width=.45\linewidth]{experiments/abc_vs_snm/all_model/snm/msimulations_model4_39.pdf}
    \label{fig:girolami_model4_snm}}
    \end{tabular}
    \caption{The simulated dynamics of the four candidate models, using
    randomly chosen subsets of parameters from the sample produced by
    SigNetMS of the power posterior distribution of $\beta = 1$, which
    is the posterior distribution of parameters, $p({\bm \theta} | M,
    {\bm D})$. Red translucent lines represent the simulated dynamics, 
    using the sampled parameters, and stronger lines represent
    overlapping simulations. Lines with blue, yellow, and green color
    represent experimental observations.}
    \label{fig:girolami_allmodel_snm}
\end{figure}

\begin{figure}[ht]
    \centering
    \begin{tabular}{c c}
    \subfigure{
    \includegraphics[clip=true,width=.45\linewidth]{experiments/abc_vs_snm/all_model/snm/msimulations_model1_0.pdf}
    \label{fig:girolami_model1_0_snm}}
    &
    \subfigure{
    \includegraphics[clip=true,width=.45\linewidth]{experiments/abc_vs_snm/all_model/snm/msimulations_model1_21.pdf}
    \label{fig:girolami_model1_1_snm}} 
    \\
    \subfigure{
    \includegraphics[clip=true,width=.45\linewidth]{experiments/abc_vs_snm/all_model/snm/msimulations_model1_25.pdf}
    \label{fig:girolami_model1_2_snm}}
&
    \subfigure{
    \includegraphics[clip=true,width=.45\linewidth]{experiments/abc_vs_snm/all_model/snm/msimulations_model1_28.pdf}
    \label{fig:girolami_model1_3_snm}}
    \\
    \subfigure{
    \includegraphics[clip=true,width=.45\linewidth]{experiments/abc_vs_snm/all_model/snm/msimulations_model1_31.pdf}
    \label{fig:girolami_model1_2_snm}}
&
    \subfigure{
    \includegraphics[clip=true,width=.45\linewidth]{experiments/abc_vs_snm/all_model/snm/msimulations_model1_39.pdf}
    \label{fig:girolami_model1_3_snm}}
    \end{tabular}
    \caption{The dynamics induced by samples of different power
    posterior distributions of parameters of the correct model. The
    value of $\beta$ increases from left to right and from top to
    bottom. We can observe how the curve of simulations progressively
    fits the curve of experimental observations.}
    \label{fig:girolami_model1_progression_snm}
\end{figure}

The ranking and the simulations indicate that SigNetMS is more
appropriate for our applications. To further investigate the results
produced by this software, we also created plots of approximations of
the produced power posterior samples, presented on 
figure~\ref{fig:girolami_model1_progression_snm}. These density function
estimates were created using the {\tt distplot} function of the Seaborn
Python package, which uses Gaussian Kernel Densinty Estimate (KDE) to
provide an estimation of the density function given a sample of such
distribution. The presented figure shows estimated power posterior
distributions, with different values of $\beta$, for the $k_1$ parameter
of the correct model, which had value $0.07$ when artificial
experimental data was created. We can see that as we increase the value 
of $\beta$, the posterior distribution concentrates on values around the
``true'' value of the parameter. It is also interesting to see that
although the estimated posterior is not necessarily centered on the 
``true'' value, the created simulations do approximate the experimental
observations.

\begin{figure}[ht]
    \centering
    \begin{tabular}{c c}
    \subfigure{
        \includegraphics[clip=true,width=.45\linewidth]{experiments/abc_vs_snm/parameters_snm/model1_0_p0_k_1.pdf}
    \label{fig:girolami_model1_0_parameters}}
    &
    \subfigure{
    \includegraphics[clip=true,width=.45\linewidth]{experiments/abc_vs_snm/parameters_snm/model1_21_p0_k_1.pdf}
    \label{fig:girolami_model1_1_parameters}} 
    \\
    \subfigure{
    \includegraphics[clip=true,width=.45\linewidth]{experiments/abc_vs_snm/parameters_snm/model1_25_p0_k_1.pdf}
    \label{fig:girolami_model1_2_parameters}}
&
    \subfigure{
    \includegraphics[clip=true,width=.45\linewidth]{experiments/abc_vs_snm/parameters_snm/model1_28_p0_k_1.pdf}
    \label{fig:girolami_model1_3_parameters}}
    \\
    \subfigure{
    \includegraphics[clip=true,width=.45\linewidth]{experiments/abc_vs_snm/parameters_snm/model1_31_p0_k_1.pdf}
    \label{fig:girolami_model1_2_parameters}}
&
    \subfigure{
    \includegraphics[clip=true,width=.45\linewidth]{experiments/abc_vs_snm/parameters_snm/model1_39_p0_k_1.pdf}
    \label{fig:girolami_model1_parameters}}
    \end{tabular}
    \caption{Approximation of the power posterior distribution of the 
    samples created by SigNetMS of parameter $k_1$ of the correct model.
    We show on this graph the estimated distribution of six different
    power posteriors, with increasing value of $\beta$ from left to
    right, from top to bottom. The approximation of this graph is
    created using {\tt distplot} function of Seaborn package. The value
    used for this parameter on the creation of the experimental data is
    $0.07$, and it is represented on the axis of the plots.}
    \label{fig:girolami_model1_progression_snm}
\end{figure}

\clearpage
\section{Model selection as a feature selection problem}
% After defining that SigNetMS is our software choice for model
% selection, we will construct another model selection problem instance.
% This time, we intend to use such instance to evaluate the approach
% of solving a model selection problem as a feature selection problem,
% with a Bayesian cost function. Although the instance we use is still 
% small, we will be able to access the surface of possible solutions and 
% get hints of how this surface is induced by the cost function we
% chose.
After defining that SigNetMS is our software choice for model selection,
we are now able to experiment and analyze the approach of solving a 
model selection problem as a feature selection problem, using a Bayesian
approach to define a cost function. To accomplish this, we created
another simple instance of model selection. Although this instance is
still a toy model, we will be able to access the space of possible
solutions and get a glance of the cost surface induced by SigNetMS over
this space.

A feature selection instance can be defined by a pair $(S, c)$ where $S$
is a set of features and $c$ is a cost function that evaluates subsets
of $S$. The space of solution is usually the power set of $S$,
$\mathcal{P}(S)$, and the cost function usually takes values from this 
space to positive real values, $c: \mathcal{P}(S) \to \fieldR$. An
optimal solution is a subset $X \in \mathcal{P}(S)$ such that $c(X) \leq
c(Y), \forall Y \in \powerset (S)$, however, it is important to notice
that the size of the search space grows exponentially with the number of
features and, in practice, with time consuming cost functions, it is
computationally unfeasible use optimal search algorithms. That is
exactly our case, since the cost function we propose to use, based on
SigNetMS, depends on the estimation of multiple power posteriors of
parameters and includes numerous numerical integrations of a system of
ordinary differential equations.

It is often useful to represent the search space with a boolean lattice, 
which is defined by the power set $\mathcal{P}(S)$ and the partial order 
relation $\subseteq$. More than an aid to represent the search space,
the boolean lattice provides a structure that is useful for search
algorithms to define paths and to take advantage of surface of the 
search space, as it is done on algorithms for the U-Curve problem, a 
special case of the feature selection problem where the cost function 
describes a u-shaped curve on every chain of the boolean lattice. The
U-Curve problem is still an NP-hard problem as the feature selection
problem is~\cite{Rei12}, however there are heuristics and optimal
algorithms that can be used on the U-Curve problem to produce a quality
answer (optimal or close to be optimal) with a feasible computational
time. Moreover, one can produce good results using U-Curve algorithms to 
solve a feature selection instance that is not necessarily U-Curve, but
does reproduce u-shaped curves with a few oscillations on chains of the
search space.

In our application, we are going to convert a model selection problem
into a feature selection problem. To do so, we define a set of candidate
reactions $S$ and a base model. Then, we consider that a set of feature
$X \in \powerset (S)$ represents a candidate model composed by the base
model plus the reactions from $X$. The cost function we use is the
logarithm of the marginal likelihood produced by SigNetMS.
Figure~\ref{fig:feature_selection_model_selection} shows an example of
feature selection instance that represents a model selection instance.

\begin{figure}[h!]
\begin{center}
    \includegraphics[width=1\textwidth]{experiments/Boolean_lattice_model_selection.pdf}
    \caption{A diagram with the search space and costs of a feature 
    selection problem that represents a model selection problem. Each 
    rectangle represents a subset of the power set of features, were
    reactions from such subset are drawn with dots. Links between 
    rectangles represent the $\subseteq$ relation; that is, two 
    rectangles linked represent two subsets of reactions $X$ and $Y$ 
    such that $X \subseteq Y$ (or $Y \subseteq X$). The set of features 
    is composed by three reactions, $\{a, b, c\}$, inducing the search 
    space with eight elements. The base model is composed by two 
    chemical species, $1$ and $2$ and a reaction between those species, 
    we can see this model at the base of the diagram, on the rectangle 
    that represents the empty set (no reaction are drawn with dots). 
    Each rectangle also represents a candidate model, which is composed 
    by the base model plus the reactions drawn with dots. The number
    below each rectangle represents the score (minus one times the cost)
    of each model. The optimal subset is drawn in yellow and it the
    subset of reactions $\{b, c\}$. Note that this instance is a U-Curve 
    instance, if we consider the cost as minus one times the score: for 
    every chain of rectangles, the cost of the model describes a u 
    shaped curve. As an instance, consider the chain $\emptyset, \{c\} 
    , \{a, c\}, \{a, b, c\}$, which has respectively the costs of
    $-0.10$, $-0.15$, $-0.30$, and $-0.03$.}
    \label{fig:feature_selection_model_selection}
    \end{center}
\end{figure}

After defining the feature selection instance, we will traverse the
search space in two ways. First, we will traverse chains from the empty
set to the complete set, to understand the shape of the cost function as
we increase the number of reactions. Then, we will run the Sequential 
Forward Search algorithm, a heuristic for the feature selection problem,
to try to find a good solution for our model selection problem.


% talk about search algorithms
% talk about the U-Curve problem  

% introduce an example of the boolean lattice

% The feature selection problem is...
% the features are blah,
% the cost function is bleh,

\subsection{Defining the feature selection instance}
% The instance we chose to show is a Ras switch pathway. This pathway...
% This time we also define the correct model, which is shown on Figure.
% We define the space of features as the set of reactions shown on
% Figure.

% We define as a "base model", a simple pathway which has no reactions,
% and for each node of the search space, we consider that such node
% represents the empty model with the reactions present on that node.

% The list of candidate reactions is shown on figure blah
% 

The instance we prepared is based on a Ras switch pathway. Ras
represents a family of proteins that are common on signaling pathways
that participate on cell growth and differentiation. Because of this
participation, Ras proteins that are constantly switched on can play a 
part on some types of cancer. The model we consider for generating 
experiments, which we also call the ``correct'' model is shown on 
Figure~\ref{fig:ras_switch:correct_model}. This model shows a pathway
that decides the state of a Ras protein as switched on, represented by
RasGTP, and as switched off, represented by RasGDP. Five other chemical
species are present on this model, and they have the following initial
concentration: 200 for SOS; 0 for SOS\_allo\_RasGDP and
SOS\_allo\_RasGTP; 900 for RasGDP; 100 for RasGTP; 200 for GEF; and 125
for GAP. Once again, we omit concentration and reaction rate units for
the sake of simplificty. These chemical species interact in eight
different chemical reactions:
\begin{itemize}
    \item{$\ce{SOS + RasGDP ->[k2] SOS\_allo\_RasGDP}$};
    \item{$\ce{SOS\_allo\_RasGDP ->[d2] SOS + RasGDP}$};
    \item{$\ce{SOS + RasGTP ->[k1] SOS\_allo\_RasGTP}$};
    \item{$\ce{SOS\_allo\_RasGTP ->[d1] SOS + RasGTP}$};
    \item{$\ce{RasGTP -> RasGDP}$}, with SOS\_allo\_RasGTP as a
        catalyst, and $k3cat$ and $K3m$ as catalytic constant and
        Michaelis constant, respectively;
    \item{$\ce{RasGTP -> RasGDP}$}, with SOS\_allo\_RasGDP as a 
        catalyst, and $k4cat$ and $K4m$ as catalytic constant and
        Michaelis constant, respectively;
    \item{$\ce{RasGTP -> RasGDP}$}, with GEF as a catalyst, and
        $k6cat$ and $K6m$ as catalytic constant and Michaelis constant,
        respectively;
    \item{$\ce{RasGDP -> RasGTP}$}, with GAP as a catalyst, and
        $k5cat$ and $K5m$ as catalytic constant and Michaelis constant,
        respectively.
\end{itemize}

\begin{figure}[H]
\begin{center}
\includegraphics[width=1\textwidth]{experiments/ras_switch/correct.pdf}
\caption{A representation of a Ras switch pathway that we consider as
    the correct model for our model selection experiment. This model
    contains seven chemical species, and eight different chemical
    reactions. Reversible reactions are represented with arrows poiting
    both directions, with reaction rate parameters over the reaction,
    with the forward reaction parameter first and then the parameter of
    the reverse reaction. Michaelis-Menten reactions are represented
    with parameters to the left of the arrow from starting at the
    enzyme, with the catalystic parameter first, and then the Michaelis
    constant.
}
\label{fig:ras_switch:correct_model}
\end{center}
\end{figure}

We used this model to generate an artificial experiment where the
concentration of activated Ras was measured at the time steps of 30,
60, 90, 120, 150, 180, 210, and 240 seconds.
Figure~\ref{fig:ras_switch:experimental_observations} shows a graph of
such experimental measurements. Similarly to the experiment of the 
previous section, those observations were created by simulating the 
correct model and then adding a Gaussian error of mean zero and standard
deviation of 0.01. The reaction rate parameters used to create these
simulations are: k1 $= 1.8e-4$, d1 $= 3$, k2 $= 1.7e-4$, d2 $=0.04$,
k3cat $= 3.8$, K3m $=1.64e3$, k4cat $= 0.003$, K4m $= 9.12e3$, k5cat
$= 0.1$, K5m $= 1.07e2$, k6cat $= 0.01$, and K6m $=1836$ (once again,
remember that we are omitting units for the sake of simplicity).
%<parameter id="k1"    value="1.8e-4"/>
%<parameter id="d1"    value="3.0"/>
%<parameter id="k2"    value="1.7e-4"/>
%<parameter id="d2"    value="0.04"/>
%<parameter id="k3cat" value="3.8"/>
%<parameter id="K3m"   value="1.64e3"/>
%<parameter id="k4cat" value="0.003"/>
%<parameter id="K4m"   value="9.12e3"/>
%<parameter id="k5cat" value="0.1"/>
%<parameter id="K5m"   value="1.07e2"/>
%<parameter id="k6cat" value="0.01"/>
%<parameter id="K6m"   value="1836"/>

\begin{figure}[H]
\begin{center}
\includegraphics[width=.75\textwidth]{experiments/ras_switch/experiment_plot.pdf}
\caption{Experimental data generated with our correct model for the Ras
    switch pathway. There is a set of three observations being
    represented, however, the time points observations of [RasGTP] are
    close enough that the lines of each experiment overlap. The time
    points considered are 30, 60, 90, 120, 150, 180, 210, and 240
    seconds; the lines are produced with a linear interpolation of
    measurements on these points.}
\label{fig:ras_switch:experimental_observations}
\end{center}
\end{figure}

To construct our search space, we considered as the base of our search
space a trivial model, containing no reactions. For the set of features,
we considered all reactions from the correct model (i.e. the correct
model is present on the search space), plus two other reactions. A
complete list of candidate reactions is shown of
Figure~\ref{fig:ras_switch:features}. The complete list of candidate
reactions is saved in a JSON file, which also stores information about
rate parameters for each reaction, including its name and prior. In this
experiment we defined Uniform prior distributions, using our prior
knowledge about reactions. As a matter of fact, in our first approach we
used Gamma distributions, however, due to numerical instabilities we
decided Uniform distributions for its simplicity and restrictiveness.

Figure~\ref{fig:ras_switch:features} also presents an arbitrary order we
choose for these reactions. This order, from (a) to (j) is useful to
enumerate points of the search space, something that search algorithms
can rely on when traversing the search space. With this order, we can
introduce the concept of characteristic vector, which is a vector
composed by boolean flags that represents a point of the search space,
and, that is, a subset of features. If the i-th element of the
characteristic vector is a 0, then the i-th feature is not present on 
the subset represented, if the i-th elemnt is 1, then this feature is
present on the subset. Considering the order we choose for this Ras
switch instance, we could say that the subset represented by the
characteristic vector 0101010001 has the reactions: SOS\_allo\_RasGDP
decomplexation, SOS\_allo\_RasGTP decomplexation, Ras activation by
SOS\_allo\_RasGTP, and GAP activation by RasGTP.

After defining our set of features, a base model, the rule for 
defining the model associated to a subset of features, and the cost
function, we are now able to experiment solving the model selection
problem as a feature selection problem. It is important to remember that
our cost function, the output of SigNetMS, is computationally expensive
and, since we have 10 features, the search space has the size $2^{10}$.
That makes it unfeasible to optimally search on the space, and therefore
we choose a heuristic to search for a good solution.


\begin{figure}[H]
    \centering
    \begin{tabular}{cc}
    \subfigure[SOS\_allo\_RasGDP complexation]{
        \includegraphics[clip=true,width=.35\linewidth]{experiments/ras_switch/reactions/sos_allo_rasgdp_complexation.pdf}
    }
    &
    \subfigure[SOS\_allo\_RasGDP decomplexation]{
        \includegraphics[clip=true,width=.35\linewidth]{experiments/ras_switch/reactions/sos_allo_rasgdp_decomplexation.pdf}
    }
    \\
    \subfigure[SOS\_allo\_RasGTP complexation]{
    \includegraphics[clip=true,width=.35\linewidth]{experiments/ras_switch/reactions/sos_allo_rasgtp_complexation.pdf}
    }
    &
    \subfigure[SOS\_allo\_RasGTP decomplexation]{
    \includegraphics[clip=true,width=.35\linewidth]{experiments/ras_switch/reactions/sos_allo_rasgtp_decomplexation.pdf}
    }
    \\
    \subfigure[Ras inactivation by GAP]{
    \includegraphics[clip=true,width=.35\linewidth]{experiments/ras_switch/reactions/ras_inactivation_by_gap.pdf}
    }
    &
    \subfigure[Ras activation by SOS\_allo\_RasGTP]{
    \includegraphics[clip=true,width=.35\linewidth]{experiments/ras_switch/reactions/ras_activation_by_sos_allo_RasGTP.pdf}
    }
    \\
    \subfigure[Ras activation by SOS\_allo\_RasGDP]{
    \includegraphics[clip=true,width=.35\linewidth]{experiments/ras_switch/reactions/ras_activation_by_sos_allo_RasGDP.pdf}
    }
    &
    \subfigure[Ras activation by GEF]{
    \includegraphics[clip=true,width=.35\linewidth]{experiments/ras_switch/reactions/ras_activation_by_GEF.pdf}
    }
    \\
    \subfigure[Ras activation by SOS]{
    \includegraphics[clip=true,width=.35\linewidth]{experiments/ras_switch/reactions/ras_activation_by_SOS.pdf}
    }
    &
    \subfigure[GAP activation by RasGTP]{
    \includegraphics[clip=true,width=.35\linewidth]{experiments/ras_switch/reactions/gap_activation_by_RasGTP.pdf}
    }
    \end{tabular}
    \caption{All candidate reactions, or the set of features for our
    feature selection problem. Reactions (a) to (h) are all the
    reactions of the considered ``correct'' model.}
    \label{fig:ras_switch:features}
\end{figure}

% SFS experiment
\subsection{Using Sequential Forward Selection to select a model}
As discussed in the previous section, the feature selection instance we
have in hands has a search space and cost function that makes it
unfeasible to use an optimal algorithm of feature selection. Therefore,
we use the Sequential Forward Selection (SFS)~\cite{Whitney1971} 
heuristics to search for a solution.
% also want to talk about what we want to check in this experiment.
% With this experiment we will be able to analyze the solution given in
% our approach. We expect that we are able to at least find a model that
% correctly describes the dynamics observed on the experiments

The Sequential Forward Selection heuristics is a greedy algorithm that
starts with the empty set of features and then, for each iteration,
decides to add the ``best'' remaining feature to the current solution.
Although it is very easy to find instances in which this algorithm is
not optimal (including U-Curve instances), it can generally find  fairly
good solutions in many cases, making it a good choice to explore spaces
with unknown surfaces, such as in our case. Pseudocode~\ref{code:sfs}
presents the dynamics of the SFS heuristics.

% pseudo código
\begin{algorithm}[H]
\textsc{SFS} $(S, c)$
\begin{algorithmic}[1]
    \State $X \gets \emptyset$
    \State {\tt did\_impove} $\gets$ {\tt True}
    \While{{\tt did\_improve}}
        \State $s^* \gets NIL$
        \For{$s \in S$ and $s \notin X$}
            \If{$c(X \cup \{s\}) < c(X \cup \{s^*\})$}
                \State $s^* \gets s$
            \EndIf
        \EndFor
        \If{$s^*$ is $NIL$}
            \State {\tt did\_improve} $\gets$ {\tt False}
        \EndIf
    \EndWhile
    \Return $X$
\end{algorithmic}
\caption{A pseudocode representing the SFS algorithm.}
\label{code:sfs}
\end{algorithm}


% resultados e análise dos resultados
\subsubsection{Results of the search}
The SFS search was conduced in a server running with an Intel 
Xeon E5-2690 CPU, and 252GB of RAM memory. The total time to conduce the
experiment was about 26 hours, with 42 calls to the cost function, that
is, 42 points of the search space were visited. The chain of subsets the
algorithm traversed, from the base to the found solution, is shown on 
Table~\ref{tab:sfs_trace}.
% how long did it take do finish?
% want to talk about the chain produced

The best model found has the characteristic vector $0111011000$ and had 
the logarithm of the marginal likelihood of $7.9$. This model has the 
following reactions: SOS\_allo\_RasGTP complexation, SOS\_allo\_RasGTP
decomplexation, Ras activation by SOS\_allo\_RasGTP, Ras activation
by SOS, and Ras activation by SOS\_allo\_RasGDP.


\begin{table}[H]
\centering
\begin{tabular}{c|cll}
\hline
Characteristic Vector && \multicolumn{1}{l}{Score} &
\multicolumn{1}{l}{Cost function time (seconds)} \\
\hline
    0000000000 && 330721.05	& 851.3	    \\
    0010000000 && 245681.93	& 1083.4	\\
    0010010000 && 211.62	& 4257.4	\\
    0011010000 && -1.32	    & 5007.71	\\
    0011011000 && -4.27	    & 4458.7	\\  
    0111011000 && -7.90	    & 5035.7	\\  
\hline
\hline
\end{tabular}
\caption{The trace of the SFS run on the Ras switch instance. The
    characteristic vector determines the set of reactions present on the
    model; the Score is minus one times the marginal likelihood of
    the model; and, the Cost function time is the total time used by 
    SigNetMS to calculate the marginal likelihood.}
\label{tab:sfs_trace}
\end{table}

Even though the resulting subset is not the correct model, the marginal
likelihood indicates that the model reasonably approximates the
experimental measurements. To verify this, we took the sample of the
posterior of model parameters produced by SigNetMS and created plots of 
simulations using those sampled parameters.
Figure~\ref{fig:ras_switch_solution} shows simulations created using
different power posterior samples. We can see that as $\beta$ increases,
the simulation approximates better the experimental results, to the
point where the simulation curves overlap with the experimental curves.
That indicates that, in fact, model $0111011000$ is able to reproduce
the dynamics observed on experimental observations.

\begin{figure}[ht]
    \centering
    \begin{tabular}{c c}
    \subfigure{
    \includegraphics[clip=true,width=.45\linewidth]{experiments/ras_switch/simulations/msimulations_model_0111011000_0.pdf}}
    &
    \subfigure{
    \includegraphics[clip=true,width=.45\linewidth]{experiments/ras_switch/simulations/msimulations_model_0111011000_10.pdf}}
    \\
    \subfigure{
    \includegraphics[clip=true,width=.45\linewidth]{experiments/ras_switch/simulations/msimulations_model_0111011000_20.pdf}}
    &
    \subfigure{
    \includegraphics[clip=true,width=.45\linewidth]{experiments/ras_switch/simulations/msimulations_model_0111011000_39.pdf}}
    \\
    \end{tabular}
    \caption{Simulations created with power posterior samples of Ras
    switch model with characteristic vector $0111011000$, which is the
    model found by the SFS search. The model parameters used to create
    these simulations were sampled when the SFS algorithm evaluated the
    cost function for the model; when the cost function was called,
    SigNetMS produced both the estimation of the log of the marginal
    likelihood and samples of different power posteriors of model
    parameters. There were 40 different power posteriors sampled, and we
    show only four of them on this figure. Since the range of the plot
    area is fixed, it may seem that the first figure has less lines than
    the other ones; that only indicate that some simulations did not
    appear on the plot area.}
    \label{fig:ras_switch_solution}
\end{figure}

Since the found model is not the correct model, we decided to also
evaluate the cost function for the correct model and compare its
cost against the found model. As the result, we got that the logarithm
of the marginal likelihood of the correct model was $-62$, compared
to $10.8$ for the found model. To understand these numbers, we also
plotted the simulations found for the correct model.
Figure~\ref{fig:ras_switch_correct} shows the simulations created using
the sampled parameters for the correct model; it is possible to see that
an approximation of the experimental data was possible. However, due to
model complexity, the marginal likelihood of the correct model was 
smaller than the marginal likelihood of the found model.

As explained by Bishop~\cite{bishop2006pattern}, marginal likelihood
methods tend to select intermediate complexity models, penalizing more
complex models. This is the reason why the model $0111011000$ was
preferred over the ``correct'' model. A simplified explanation to why 
this happens on marginal likelihood methods is that more complex models
are able to generate a wider range of output data, and therefore, the
likelihood function $p({\mathcal {D}} | M_{complex})$ is spread over 
a wider space, while the mass of a simpler model is concentrated in a 
less spread area. We can argue that, because of such spreadness 
phenomena, we expect a tendency to observe 
$p({\mathcal {D} = \bm{D}} | M_{complex}) >$
$p({\mathcal {D} = \bm{D}} | M_{simple})$.

%Although we are not focusing on a machine learning view in this project,
%we could also explain this phenomena with the bias variance
%decomposition.

\begin{figure}[ht]
    \centering
    \begin{tabular}{c c}
    \subfigure{
    \includegraphics[clip=true,width=.45\linewidth]{experiments/ras_switch/simulations/msimulations_model_1111111100_0.pdf}}
    &
    \subfigure{
    \includegraphics[clip=true,width=.45\linewidth]{experiments/ras_switch/simulations/msimulations_model_1111111100_10.pdf}}
    \\
    \subfigure{
    \includegraphics[clip=true,width=.45\linewidth]{experiments/ras_switch/simulations/msimulations_model_1111111100_20.pdf}}
    &
    \subfigure{
    \includegraphics[clip=true,width=.45\linewidth]{experiments/ras_switch/simulations/msimulations_model_1111111100_39.pdf}}
    \\
    \end{tabular}
    \caption{Simulations created with power posterior samples of Ras
    switch model with characteristic vector $1111111100$, which is the
    ``correct'' model, that is, the model we used to create artificial
    experimental data. It is possible to see that for greater values of
    $\beta$, the simulations are close to experimental data.}
    \label{fig:ras_switch_correct}
\end{figure}



\chapter{Conclusion}
\label{chap:conclusion}
We will start this chapter with a review of the content presented in 
this dissertation, with some extra discussion of a few specific topics.
After that, we will present the main contributions of this work,
including technological tools and publications. Finally, we will close
this chapter with possibilities of future work related to this
dissertation.


\section{Review of Contents of this Dissertation}
% Review of the contents of this work
On the Introduction of this text, we presented the main aspects of cell
signaling pathways and how computational models can approximate their
dynamics. After that, we showed how Wu~\cite{Wu15} defined an approach
for model selection, of cell signaling pathwyas, as a feature selection
problem, and also the main caveats of their approach. With that, we
could state the goal of this project, which is to study and develop a
method for model selection using feature selection, where the cost 
function uses a Bayesian approach to calculate the cost (or score) of a
model.

% Why did we go for that goal?
% hability to input prior knowledge
% auto-penalize complex models

% Ok, then on chapter 2 we revised some content

% Then we reviewed state of the art model selection

% Then we produced an almost efficient way to calculate marginal
% likelihoods

% We then compared ABC-SysBio to SigNetMS

% And finally, we experimented on a small instance

% And what did we learn after all?


\section{Contributions of this Dissertation}
% Contributions of this work
% -> technological
% -> participation in congresses where we presented this work
%   -> Rocky 2019
%   -> São Paulo School of Data Science
% -> advanced school of mathematics

% Future work


\newpage
\printbibliography

\end{document}

