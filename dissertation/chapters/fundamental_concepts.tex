%begin-include

% What I want to talk about in this section?
% - Cells signaling pathways are part of the cell communication system,
% and it allows the cell to perceive the conditions of the environment 
% and also to change its behaviour according to the input signal.
% - The signals perceived by a cell can come from cells that very close
% (including itself), as in synapses or it can travel long distances in 
% the organism, as in hormones.
% - When a signal reaches a cell, it can either activate a recepter in
% the cell membrane or diffuse into the cell.
% - Once this happen, the signal or the activated receptor can trigger
% a sequence of chemical interaction, altering the conformation and 
% state of proteins and also changing the concentration of chemical 
% species in the cell. Ultimately, this chain of effects can alter the 
% behaviour of the cell, what is called signal transduction.
% - Cells have evolved to respond differently to different types of 
% signals present in the cell.
\section{Cell Signaling Pathways}
Cell signaling pathways are part of the complex cell communication 
system, and it allows the cell to perceive the conditions of the 
environment in which it is placed and change its behaviour accordingly.
Signaling pathways participate in the regulation of many cell functions,
including development, division and cell death. Bad functioning 
signaling pathways can also be related to diseases, as in many cases of
cancer.

The signal perceived by a cell can come from cells that are close 
(including the same cell that produced the signal), as in synapses, or 
it can travel long distances in the organism, as in hormones. When a 
signal reaches a cell, it can either penetrate the cell or bind to some 
specific receptor in the membrane. Once either of those events happen, 
the signal or the receptor can trigger a sequence of chemical 
interactions that can include the change of conformation of proteins, 
activation or inactivation of proteins, and change the concentration of
chemical species in the cell. Ultimately, this chain of chemical 
reactions caused by the signal can alter the behaviour of the cell, what 
is called signal transduction.

\section{Measurements of Proteins in Cell Signaling Pathways}
\section{Dynamic Modeling of Cell Signaling Pathways}
\section{Identification of Cell Signaling Pathways}
\section{State of the Art in Model Selection}
%\section{Bayesian Inference}
%\subsection{Girolami (bioinformatics)}
%\subsection{Kolch}
%\subsection{ABC-Sysbio}
