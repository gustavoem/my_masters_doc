%begin-include

% What I want to talk about in this section?
% - Cells signaling pathways are part of the cell communication system,
% and it allows the cell to perceive the conditions of the environment 
% and also to change its behaviour according to the input signal.
% - The signals perceived by a cell can come from cells that very close
% (including itself), as in synapses or it can travel long distances in 
% the organism, as in hormones.
% - When a signal reaches a cell, it can either activate a recepter in
% the cell membrane or diffuse into the cell.
% - Once this happen, the signal or the activated receptor can trigger
% a sequence of chemical interaction, altering the conformation and 
% state of proteins and also changing the concentration of chemical 
% species in the cell. Ultimately, this chain of effects can alter the 
% behaviour of the cell, what is called signal transduction.
% - Studying signaling pathways is important because they help us 
% understand the mechanisms of a cell, which can for example, elucidate 
% treatments for diseases.
\section{Cell Signaling Pathways}
Cell signaling pathways are part of the complex cell communication 
system, and it allows the cell to perceive the conditions of the 
environment in which it is placed and change its behaviour accordingly.
Signaling pathways participate in the regulation of many cell functions,
including development, division and cell death. Bad functioning 
signaling pathways can also be related to diseases, as in many cases of
cancer.

The signal perceived by a cell can come from cells that are close 
(including the same cell that produced the signal), as in synapses, or 
it can travel long distances in the organism, as in hormones. When a 
signal reaches a cell, it can either penetrate the cell or bind to some 
specific receptor in the membrane. Once either of those events happen, 
the signal or the receptor can trigger a sequence of chemical 
interactions that can include the change of conformation of proteins, 
activation or inactivation of proteins, and change the concentration of
chemical species in the cell. Ultimately, this chain of chemical 
reactions caused by the signal can alter the behaviour of the cell, what 
is called signal transduction.

Since signaling pathways participate in many of the cell functions, and 
might even be related to diseases, it is important to study those 
structures, in order to get a better understanding of the cell 
mechanisms and diseases. One approach on the study of the cell signaling
pathways is to measure the concentration change of proteins that 
participate on the pathway of interest.

% - Western blot is a technique that can indicate the amount of a 
% specific protein that is in a mixture of proteins
% - This technique shows the presence of a protein in a mixture by 
% ``blotting'' these molecules into a membrane. 
% - Repeating the procedure in different times define time-course 
% observations of the protein in a biological experiment.
% - With these observations in hand, the researcher is able to construct 
% a measurement that is relevant to describe the biological experiment. 
% For instance, in a signaling network where a protein is closely 
% related to the behaviour of interest of the cell, this protein is a 
% good candidate as a measure of the system.
\section{Measurements of Proteins in Cell Signaling Pathways}
Western blot is a laboratory technique that can indicate the amount of a
specific protein that is present in a mixture of proteins. This 
technique show the presence of a protein in a mixture by ``blotting'' 
the membrane where the molecules of interest are located. We can 
superficially summarize the procedure in the following steps: first a 
mixture with a group of cells of interest must be created; second, 
proteins from the mixture should be fixed on the blotting membrane; 
third, an antibody should bind to the target protein molecules; and 
finally, a method for colorizing the antibody should be applied. An 
image of the resulting membrane can then be analyzed with computer
programs to quantify the relative concentration of the protein of 
interest.

Repeating this procedure in different times define time-course 
observations of the protein in the biological experiment. With these
observations in hand, a researcher might be able to construct a 
measurement that can summarize the experiment; for instance, in a 
signaling network where a protein is closely related to the behaviour of
interest on the cell, this protein concentration is a good candidate for
a measurement that nearly explains the experiment.

\section{Dynamic Modeling of Cell Signaling Pathways}
\section{Identification of Cell Signaling Pathways}
\section{State of the Art in Model Selection}
%\section{Bayesian Inference}
%\subsection{Girolami (bioinformatics)}
%\subsection{Kolch}
%\subsection{ABC-Sysbio}
