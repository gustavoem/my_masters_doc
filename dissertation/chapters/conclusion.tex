We will start this chapter with a review of the content presented in 
this dissertation, with some extra discussion of a few specific topics.
After that, we will present the main contributions of this work,
including technological tools and publications. Finally, we will close
this chapter with possibilities of future work related to this
dissertation.


\section{Review of Contents of this Dissertation}
% Review of the contents of this work
On the Introduction of this text, we presented the main aspects of cell
signaling pathways and how computational models can approximate their
dynamics. After that, we showed how Wu~\cite{Wu15} defined an approach
for model selection of cell signaling pathwyas as a feature selection
problem, and also the main caveats of their approach. With that, we
could state the goal of this project, which is to study and develop a
similar method for model selection using feature selection, where the 
cost function is created with a Bayesian approach, capable of
determining the likelihood of a model producing experimental data.

% Why did we go for that goal?
% hability to input prior knowledge
% auto-penalize complex models

We decided to use a Bayesian approach to construct the cost function
because of its ability to auto penalize overly complex models, avoiding
some sort of ``overfit'' that can occur, characterized by complex models
that have good ability to fit an example of experimental data, but have 
poor generalization performance. Another reason to use a Bayesian
approach is that this approach consider model parameters as random 
variables, which allow the user to input their prior knowledge, and also
to produce a posterior distribution for model parameters. Specially in 
our application, where model parameters are related to the rate 
constants of reactions, using a non deterministic approach to model 
constants is closer to what is seen in the real world; since those 
reactions can occur in a variety of conditions, reaction rate constants
are not fixed constants.

% Ok, then on chapter 2 we revised some content
Before introducing concepts and methodologies we developed in this work,
we reviewed the fundamental concepts necessary to advance in this 
project. On Chapter 2 we conduced this review, presenting concepts
of cell signaling pathways, experimental measurements of those systems, 
and also how to create computational models for them, using system of 
ordinary differential equations. We also introduced the state of the art
methodologies for model selection in the context of cell signaling
pathways and the algorithm of Metropolis-Hastings, to generate samples
of unknown distributions.

% Then we reviewed state of the art model selection
On Chapter 3 we presented a short review of two different methodologies
for evaluating the quality of models. The first methodology, uses an
estimative of the marginal likelihood of a model being the ``correct''
given the experimental data. This estimative is created by sampling from
different power posterior distributions, bridging the prior and
posterior distribution of model parameters; to conduce those
calculations, it is necessary to define a likelihood function. The 
second methodology, uses the concept of Approximate Bayesian 
Computation, and it also produces a sequence samples that for each
iteration approximates better the posterior distribution of parameters.
This last approach however, does not depend on the definition of a
likelihood function. There are available software that apply both
methodology, BioBayes and ABC-SysBio, for the first and second
approaches, respectively.

% Then we produced an almost efficient way to calculate marginal
% likelihoods
After testing both software, BioBayes and ABC-SysBio, we found that
BioBayes would be cumbersome to use as part of a feature selection cost
function. We then decided to produce a new software to estimate marginal
likelihood of models, and we called it SigNetMS. On Chapter~4 we
presented the main procedures we needed to build for this software. We
highlighted the sampling procedure as a computationally expensive
procedure, mainly because of the multiple numerical integrations
necessary to build the sample. We discussed how we decided for a
numerical integration software, since producing one ourselves was not
possible in the scope of this project. We discussed how we could
optimize our process in order to make more efficient calls of the
integrator, using symbolic mathematics to represent our system. Finally,
we presented how we managed to sample from multiple power posterior
distributions, using parallelization.

% We then compared ABC-SysBio to SigNetMS
With the implementation of SigNetMS, we could then compare both
methodologies to evaluate models: based on the marginal likelihood
estimation, using SigNetMS, and based on Approximate Bayesian
Computation, using ABC-SysBio. To compare them, we used a simple model
selection instance, with four models including a ``correct'' model, used
to generate artificial experimental measurements. As the result we
could see that SigNetMS produced a better ranking of models compared to
ABC-SysBio. We could also see that SigNetMS evaluated a simpler model as
better than a model with similar ability to reproduce experimental data,
but with higher complexity; which is an evidence that, in fact, the
Bayesian approach does penalize the complexity of models.

% And finally, we experimented on a small instance
Since SigNetMS produced better results on the simple model selection
instance, we used this software in our next experiment: a model
selection problem solved as a feature selection problem. To build this
instance, we defined a ``correct'' model of a Ras switch to generate
artificial experimental data, and a set of candidate reactions
(including all reactions of the correct model). Then, we created a
feature selection instance where the set of features is the set of
candidate reactions, and the cost of a subset is minus one times the
marginal likelihood of a model with such subset. To conduce the search,
we used the Sequential Forward Selection heuristics, and as a result, we
got a subset with five features, different from the ``correct'' subset,
with eight features. We produced plots that indicate that for both
``correct'' and found subset, the produced sample does make the model
reproduce the experimental data; however, the cost of the ``correct''
model was higher, which indicates that its complexity was penalized,
since a simpler model could reproduce its measured dynamics.

% And what did we learn after all?
Finally, with this last experiment, we could actually confirm that the
Bayesian approach does penalize complex models. More than that, we could
confirm that the sample produced by SigNetMS, when possible, does
approximate the model dynamics to the experimental measurements
dynamics. However this results indicates that SigNetMS could be used for
other examples of model selection, we should also highlight that our
experiment was small, and further experimentation is needed to improve
the SigNetMS software, preparing it for larger and harder instances.

\section{Contributions of this Dissertation}
% Contributions of this work
% -> technological
% -> participation in congresses where we presented this work
%   -> Rocky 2019
%   -> São Paulo School of Data Science
% -> advanced school of mathematics
The main contributions during this work are the following:

\begin{itemize}
\item{The implementation of SigNetMS, a free software that allows
    evaluating the quality of models, generating an estimative of the
    marginal likelihood of a model, and also a posterior sample of model
    parameters.}
\item{The comparison of two Bayesian approaches for model selection:
    using marginal likelihood and approximate Bayesian computation.
    Conduced on Chapter~\ref{chap:experiments}.}
\item{The experimentation of a feature selection approach for model
    selection, using a Bayesian cost function, conduced on
    Chapter~\ref{chap:experiments}.}
\item{The participation at the São Paulo School of Advanced Science on 
    Learning from Data, as a volunteer, including a five minute
    lightning talk about this work.}
\item{The participation at the 2019 Rocky Mountain Bioinformatics 
    Conference, including a poster session where this work was
    presented.}
\end{itemize}

\section{Suggestions for Future Work}
% Future work
For future work related to this project, we would like to recommend the
following topics for further research:
\begin{itemize}
    \item{An efficiency improvement for SigNetMS.} The sampling of model
        parameters is still a very time consuming procedure, and to
        improve its performance, we would like to suggest a few options:
        further research on numerical integration software, possibly
        aiming for software that are optimized for parallelization and
        memoization, since many integrations are performed.
    \item{Treatment for numerical instability on the integration of
        models. For many of our instances, we noticed that some regions
        of the space of parameters can lead to numerical instabilities
        in the integration methods. It would be necessary to investigate
        on how to avoid such areas and what is the influence of such
        avoidance in the produced sample.}
    %solving the problem as it was u-curve
    \item{Solving the feature selection problem as a U-Curve problem. In
        this work we produced evidences that complex models are
        penalized, and therefore, it would be interesting to analyze if
        the U-Curve simplification is applicable for model selection
        instances.}
    %heterogeneous conditions of measurements
    \item{Experimentation on heterogeneous conditions of measurement. In
        all the instances we used in this project, repetitions of 
        experimental measurements had very similar values over time; it
        might be interesting to assess the robustness of our methodology
        when experimental measurements are taken in more heterogeneous
        conditions, with variations in the observed dynamics.}
    %application of the methodology on real examples
    \item{Applications of the methodology on real instances of model
        selection. In this project, although inspired in real signaling
        network pathways, we used only artificial examples. It is
        therefore important to use this methodology in real instances,
        where the ``correct'' model is unknown.}
\end{itemize}
