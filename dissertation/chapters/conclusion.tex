We will start this chapter with a review of the content presented in 
this dissertation, with some extra discussion of a few specific topics.
After that, we will present the main contributions of this work,
including technological tools and publications. Finally, we will close
this chapter with possibilities of future work related to this
dissertation.


\section{Review of Contents of this Dissertation}
% Review of the contents of this work
On the Introduction of this text, we presented the main aspects of cell
signaling pathways and how computational models can approximate their
dynamics. After that, we showed how Wu~\cite{Wu15} defined an approach
for model selection, of cell signaling pathwyas, as a feature selection
problem, and also the main caveats of their approach. With that, we
could state the goal of this project, which is to study and develop a
method for model selection using feature selection, where the cost 
function uses a Bayesian approach to calculate the cost (or score) of a
model.

We decided to use a Bayesian approach to construct the cost function
because of its ability to auto penalize overly complex models, avoiding
some sort of ``overfit'', characterized by complex models that have good
ability to fit an example of experimental data, but have poor
generalization performance. Moreover, Bayesian approaches consider model
constants as random variables, which allow the user to input their prior
knowledge of constants, and also to produce a posterior distribution
for model constants. Specially in our application, where model constants
are related to the speed of reactions, using a non deterministic 
approach to model constants is closer to what is seen in the real world.

% Why did we go for that goal?
% hability to input prior knowledge
% auto-penalize complex models

Before introducing concepts and methods of current works on model 
selection, we reviewed the fundamental concepts, necessary to advance in
this project. On Chapter 2 we conduced this review, presenting concepts
of cell signaling pathways, measurements of those systems, and also how
can we create computational models for them, using system of ordinary
differential equations. % Then, we...

% Ok, then on chapter 2 we revised some content

% Then we reviewed state of the art model selection

% Then we produced an almost efficient way to calculate marginal
% likelihoods

% We then compared ABC-SysBio to SigNetMS

% And finally, we experimented on a small instance

% And what did we learn after all?


\section{Contributions of this Dissertation}
% Contributions of this work
% -> technological
% -> participation in congresses where we presented this work
%   -> Rocky 2019
%   -> São Paulo School of Data Science
% -> advanced school of mathematics

% Future work
