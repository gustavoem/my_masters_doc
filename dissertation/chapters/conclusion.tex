We will start this chapter with a review of the content presented in 
this dissertation, with some extra discussion of a few specific topics.
After that, we will present the main contributions of this work,
including technological tools and publications. Finally, we will close
this chapter with possibilities of future work related to this
dissertation.


\section{Review of Contents of this Dissertation}
% Review of the contents of this work
On the Introduction of this text, we presented the main aspects of cell
signaling pathways and how computational models can approximate their
dynamics. After that, we showed how Wu~\cite{Wu15} defined an approach
for model selection of cell signaling pathwyas as a feature selection
problem, and also the main caveats of their approach. With that, we
could state the goal of this project, which is to study and develop a
similar method for model selection using feature selection, where the 
cost function is created with a Bayesian approach, capable of
determining the likelihood of a model producing experimental data.

% Why did we go for that goal?
% hability to input prior knowledge
% auto-penalize complex models

We decided to use a Bayesian approach to construct the cost function
because of its ability to auto penalize overly complex models, avoiding
some sort of ``overfit'' that can occur, characterized by complex models
that have good ability to fit an example of experimental data, but have 
poor generalization performance. Another reason to use a Bayesian
approach is that this approach consider model parameters as random 
variables, which allow the user to input their prior knowledge, and also
to produce a posterior distribution for model parameters. Specially in 
our application, where model parameters are related to the rate 
constants of reactions, using a non deterministic approach to model 
constants is closer to what is seen in the real world; since those 
reactions can occur in a variety of conditions, reaction rate constants
are not fixed constants.

% Ok, then on chapter 2 we revised some content
Before introducing concepts and methodologies we developed in this work,
we reviewed the fundamental concepts necessary to advance in this 
project. On Chapter 2 we conduced this review, presenting concepts
of cell signaling pathways, experimental measurements of those systems, 
and also how to create computational models for them, using system of 
ordinary differential equations. We also introduced the state of the art
methodologies for model selection in the context of cell signaling
pathways and the algorithm of Metropolis-Hastings, to generate samples
of unknown distributions.

% Then we reviewed state of the art model selection
On Chapter 3 we presented a short review of two different methodologies
for evaluating the quality of models. The first methodology, uses an
estimative of the marginal likelihood of a model being the ``correct''
given the experimental data. This estimative is created by sampling from
different power posterior distributions, bridging the prior and
posterior distribution of model parameters; to conduce those
calculations, it is necessary to define a likelihood function. The 
second methodology, uses the concept of Approximate Bayesian 
Computation, and it also produces a sequence samples that for each
iteration approximates better the posterior distribution of parameters.
This last approach however, does not depend on the definition of a
likelihood function. There are available software that apply both
methodology, BioBayes and ABC-SysBio, for the first and second
approaches, respectively.

% Then we produced an almost efficient way to calculate marginal
% likelihoods
After testing both software, BioBayes and ABC-SysBio, we found that
BioBayes would be cumbersome to use as part of a feature selection cost
function. We then decided to produce a new software to estimate marginal
likelihood of models, and we called it SigNetMS. On Chapter~4 we
presented the main procedures we needed to build for this software. We
highlighted the sampling procedure as a computationally expensive
procedure, mainly because of the multiple numerical integrations
necessary to build the sample. We discussed how we decided for a
numerical integration software, since producing one ourselves was not
possible in the scope of this project. We also discussed how we could
optimize our process in order to make more efficient calls of the
integrator, using symbolic mathematics to represent our system.

% We then compared ABC-SysBio to SigNetMS

% And finally, we experimented on a small instance

% And what did we learn after all?


\section{Contributions of this Dissertation}
% Contributions of this work
% -> technological
% -> participation in congresses where we presented this work
%   -> Rocky 2019
%   -> São Paulo School of Data Science
% -> advanced school of mathematics

% Future work
