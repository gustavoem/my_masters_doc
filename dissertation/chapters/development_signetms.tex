%begin-include
In this chapter we present the development process and implementation
details of the SigNetMS software, which stands for {\bf Sig}naling 
{\bf Net}work {\bf M}odel {\bf S}election. This software is capable of
producing an estimative of the marginal likelihood of a model given
experimental data, $p({\bm D} | M)$.

% What are we going to talk about in this chapter?
% - first, we have to talk about which methodology this software uses
%   -> make sure you citate BioBayes and how its cumbersome to use it
% to create a model ranking. we should state that we are using marginal
% likelihood estimates here.
%   -> make sure we specify the input and output of this software
%   -> include program arguments
% - then, we can talk about the implementation and optimizations
%   -> how did we implement the sampling? what was the used proposal
%   distribution?
%   -> fast integration of system of differential equations
%   -> parallel sampling
%   -> running the algorithm on a cluster

\section{The SigNetMS Software}
% General Aspects of the software

% The Bayesian methodology
% - SigNeMS creates an estimative of the marginal likelihood of a model
% given a set of experimental data.
% - To create this estimative, SigNetMS uses ideas of Thermodynamic
%   Integration.
% - There is a software that can carry similar simulations, called
%   BioBayes, however we found its use cumbersome, mainly because:
%   it is a GUI software, which does not fit our type of applications,
%   that should be ran in a server for several hours. Moreover, we
%   intended to link the marginal likelihood output to other programs,
%   for model selection.

% Details on how the program creates the system of ODEs
% Details on how the proram estimates the likelihood

SigNetMS is a Python program that can be used as a tool for model
selection. The source code is available on 
Github\footnote{https://github.com/gustavoem/SigNetMS} and it is open
source, under the GNU General Public License.

This program expects as the input: a signaling pathway model,
represented by a Systems Biology Markup Language
(SBML)~\cite{hucka2003systems} file, with the definition of reactions,
kinetic laws and initial concentrations of chemical species; an
Extensible Markup Language (XML) file with experimental data, including 
time series measurements of the  biological phenomena of interest;
another XML file with definitions of prior distributions of reaction
rate constants; and, finally, a set of parameter values that determine
the sampling process of model parameters. There are also optional
parameters on SigNetMS, used to control random number generator seeds,
number of execution threads, and verbose runs.

The output of the program is composed by an estimative of the marginal
likelihood of the model, given experimental data, $p({\bm D} | M)$ and a
list of parameter values $({\bm \theta}^1, {\bm \theta}^2, \ldots {\bm
\theta}^l)$ that represent a sample of the distribution $p({\bm \theta}
| M, {\bm D})$. If one simulates the model $M$ with parameter values
from the sample, we expect that, the higher the marginal likelihood, the
closer the simulation is to the experimental data.

To calculate this estimative, we use the ideas of Thermodynamic
Integration we presented on section~\ref{sec:thermodynamic_integration}.
Before implementing SigNetMS we also considered using another software,
called BioBayes~\cite{Vyshemirsky2008}, that has a very similar approach
to produce an estimative of the marignal likelihood. However, we did not
proceed to use this software because there was only a graphical user
interface version of the software, which made the process of running
instances and collecting results cumbersome. We have also tried
contacting the authors to obtain the source code, however they could not
provide us that.

\subsection{Creating an estimative of the marginal likelihood}
% Since we are using a Thermodynamic Integration  approach, we need to 
% define a likelihood function, create samples of power posterior 
% distributions and, finally define an approximation to the marginal
% likelihood.

% The likelihood function...
% The samples of power posteriors are...
% The approximation of the marginal likelihood are...
% The whole process, has the following order:
% -> all inputs are read;
%   -> The sbml model is read and transformed into a System of ODE
%       -> this model can determine a simulation, given a list of
%       parameter values and a measurement unit;
%   -> Parameter priors, experimental data are read and saved
% -> sampling of posterior starts with naive sampling
% -> then we proceed to covariance shaped burn-in
% -> then we go to populational monte carlo markov chain
% -> then we use the approximation to generate the margginal likelihood

To create estimates of the marginal likelihood using the Thermodynamic
Integration approach, we need to define a likelihood function $p({\bm
D}| M, {\bm \theta})$ and also samples power posterior distributions
$p_\beta({\bm \theta})$ for a sequence of values of $\beta$.

%-> The likelihood function is...
%-> To implement this, we must simulate the model with parameter values
%of theta.
%-> these simulations involves the numerical integration of a system of 
% ordinary differential equaitons
%-> current methods of numerical integrations are iterative and can be
% very consty depending on the size of the numerical system and also on
% the type of the system, stiff or not
The likelihood function we used is the same as we defined in
equation~\ref{eq:likelihood_multivariate}:
\begin{equation*}
    p ({\bm D} | M,{\bm \theta}) = 
        p_{\mathcal{N}_{\left(\vec{0}, \Sigma\right)}}
        (\phi (M, {\bm\theta}) - {\bm D}),
\end{equation*}
where ${\bm D}$ is the experimental measurement, $M$ is the model,
${\bm \theta}$ is a set of parameter values, $\Sigma$ is the variance of
the error of experimental observations, and $\phi$ is a function that
calculates the simulation of model $M$ with parameter values ${\bm
\theta}$.  To implement this function, we must implement the $\phi$
function, which gives the experimental measurement applied to a 
simulation of the model. The simulation of the model is created by 
deriving a system of ordinary differential equations, and then
numerically integrating this system, over the time steps defined by the 
experimental measurements. 

% Numerical integration of systems are solved using iterative methods
% stiff non stiff
% we used  third-party software to solve this problem

