%begin-include
{\color{blue} Simple description of the content of this chapter}.

\section{Model ranking using Marginal Likelihood}
% Simple explanation of this model ranking
% Describe the likelihood function 
% However, this is very hard to calculate
% It is possible to apply an Importance Sampling Estimator {cite 
% Newton and Raftery}. However, it was showed on {cite girolami again}
% that these methods do not perform well.
% Then it was proposed to use Thermodynamic Integration. Make it clear 
% that this allows us to create some estimators.
% Subsection - Thermodynamic Integration
% -> explain how to derive it
%   -> name what is the power posterior
% -> include here: how to estimate it?
% Subsection - Estimation of the Marginal Likelihood 
% -> We need to find samples of the posteriors
% -> Explain that we use the methodology of Xuan
% -> We use three Metropolis-Hastings
% -> A first burn-in step with jumps independent for each single 
% parameter. Adaptive Metropolis Hastings
% -> A second using a Variance-Covariance Matrix
% -> A third using populational MCMC
The marginal likelihood of an experiment being reproduced by a model, 
$p (D | M)$, can be used as a model ranking metric as it determines 
which model makes the experimental observations more likely to happen.
We can write the marginal likelihood as:
\begin{equation}
    p (D | M) = \int_{\Theta} p (D | M, \theta) p (\theta | M)d\theta.
\label{eq:marginal_likelihood}
\end{equation}
However, calculating this integral analitically is only possible in 
very special cases and, usually, it would depend on knowing models for 
the distributions assolicated to these probability functions, which is 
generally not possible in our case.

Even though this integral is very hard to be calculated, there are 
methods that allow us to estimate its value. A straight forward method 
for estimate this integral value is the Importance Sampling 
Estimator~\cite{Newton1993}. This method uses the Monte Carlo integral 
estimation method that estimates integrals of the form $\int g(\theta)p(\theta)d\theta$

% ABC-SysBio

% Experiments comparing both approaches
