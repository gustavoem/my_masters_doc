% Outline of this section
% First we should start with a two-page description of the project with:
% - What are cell signaling pathways
% - What types of computational models we have for these pathways and 
% which one do we use
% - What can we do with those networks and how hard is it to get one to 
% work
% - What are the basic tasks to build a pathway
% - What has Lulu done and what are the limitations of her work
% - How do we intend to surmount those limitations


% Subsection Then we should state the main objectives/challenges of our 
% work
% Subsection Then we should give a short description of the work 


% What are cell signaling pathways and how is it important to study them
% - control how cell behaves in different types of environments
% - cancer cells have bad behaviour
% - that's one reason to study signaling networks
Cell Signaling pathways are cascades of chemical interactions that 
allow the communication between the cell environment and the 
cell itself. These pathways also are able to regulate many cell 
functions, including DNA replication, cell division and cell death. We
can observe the functioning of signaling pathways as a mechanism that 
can conform the cell behaviour with signals that come from the 
environment conditions in which the cell is placed. The studies of cell 
signaling pathways can lead to determining how cells can respond to 
different stimulus; for instance, with the studies of signaling pathways
activated by a chemical species, one could determine how an unhealthy 
cell would respond to a drug containing this species.

% There are computational models for signaling networks
% - Michaelis Menten equations for chemical interactions
% - With a system of ODES we can then simulate the cell behaviour
% - However, the huge number of interactions happening in the cell makes
%   it impossible to consider everything.
% - Therefore we must know 
It's possible to construct mathematical models to represent a set of
chemical reactions and consequently a signaling network. One approach on 
the modeling of those interactions is based on the law of mass action. 
This law proposes that the rate of a chemical reaction is proportional 
to the product of reactants concentrations, i.e we can calculate the 
concentration change rate of a species in an interaction by calculating 
the product of reactants concentrations up to a multiplying constant. 
If we consider the set of interactions of a signaling pathway, we can
then come up with a system of ordinary differential equations (ODEs) 
that can model the dynamics of the concentration of each chemical 
species from the pathway. Generally, these systems are complex and 
cumbersome, if not impossible, to be solved analytically, therefore we 
resort on computational models that apply numerical methods to 
approximate solutions of these systems.


% How do we create these computational models and what should they do?
%


% We can model finding reactions and constants from the literature
% However, we might have missed some interactions or even added irrelevant ones
% To solve this, Lulu did...
% Her work however had a few limitations
% To surmount these limitations

