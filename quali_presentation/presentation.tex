\documentclass{beamer}
\usepackage[utf8]{inputenc}
\usepackage[portuguese]{babel}
\usepackage{amsmath}
\usepackage{subfigure}
\usepackage{booktabs}

\usetheme{metropolis}           % Use metropolis theme

\newtheorem{mydefinition}{Definição}

\newcommand{\foreignword}[1]{\textit{#1}}
\newcommand{\toolname}[1]{\textit{#1}}
\newcommand{\fieldR}{\mathbb{R}}
\newcommand{\powerset}{\mathcal{P}}
\newcommand{\probability}{\mathbb{P}}
\newcommand{\expectation}{\mathbb{E}}
\newcommand{\algname}[1]{\texttt{#1}}
\newcommand{\langname}[1]{\texttt{#1}}
\newcommand{\varname}[1]{\texttt{#1}}
\newcommand{\floor}[1]{\lfloor #1 \rfloor}
\newcommand{\ceil}[1]{\lceil #1 \rceil}
\newcommand{\mathsc}[1]{{\normalfont\textsc{#1}}}
\newcommand{\forest}{\mathcal{F}}
\newcommand{\pfsnode}[1]{\mathbf #1}
\newcommand{\species}[1]{\textit{#1}}
\newcommand{\gender}[1]{\textit{#1}}

\graphicspath{ {img/} }

\title{Identification of cell signaling pathways based on biochemical 
reaction kinetics repositories}
\date{May 2019}
\author{Gustavo Estrela}
\institute{Instituto de Matemática e Estatística \\ 
           Centro de Toxinas, Resposta-imune e Sinalização Celular (CeTICS) \\
           Laboratório Especial de Ciclo Celular, Instituto Butantan}
\begin{document}
\maketitle
    
%Introduction
%- Cell Signaling Pathways;
%- Modeling it mathematically;
%- Creating a model;
    %- find a set of reactions;
    %- estimate the parameters of this model;
%- Lulu Wu created a method that systematically searches for a model, 
  %modifying it incrementally
    %- we can see the problem as a feature selection problem;
    %- it didn't work out so well...
    %- she was only able to reconstruct models if the starting model was
      %already similar to the correct model
    %- this may have happened because: the database could be more nearly
      %complete; the search algorithm could be more general; her cost 
      %function could not penalize well overcomplex functions.
%- We propose then to create a new methodology that overcomes theses 
  %difficulties using a new cost function, defining a broader search 
  %space and using new search algorithms.
%- Objectives of this work.



\section{Introduction}
\begin{frame}{Cell Signaling Pathwyas}
% Antes de mais nada, vamos entender o que são vias de sinalização 
% celular. 
% A sinalização celular faz parte da comunicação das células de um 
% organismo. Um sinal pode avançar 

%\begin{itemize}
%\item{
        Cell signaling pathways are important structures that allow cells
    to respond to signals that come from its environment.
%}
\pause
%\item{

        These pathways are essential for many cell functions, including
    reproduction, growth and death.
%}
\pause
%\item{

        Understanding the functioning of cell signaling pathways is 
    important in many biological areas.
%}
%\end{itemize}
\end{frame}


\begin{frame}{Cell Signaling Pathways}
\begin{itemize}
    \item{From an input signal}
\end{itemize}
\end{frame}


\begin{frame}{Modeling Cell Singaling }
\end{frame}

\begin{frame}{}
\end{frame}
%\begin{frame}{Modelos computacionais}
%\begin{frame}{Modelos computacionais}
  %Modelos computacionais são criados para simular sistemas complexos. \\
  %\pause
  %\begin{center}
  %\alert{ 
  %$
  %\begin{aligned}
    %\text{entrada} &\longrightarrow& \text{sistema} &\longrightarrow& \text{saída} \\
    %\pause
    %\text{entrada} &\longrightarrow& \text{modelo} &\longrightarrow& \sim\text{saída} \\
  %\end{aligned}
  %$
  %}
  %\end{center}
%\end{frame} 

\begin{frame}{}
\end{frame}

\begin{frame}{}
\end{frame}

\begin{frame}{}
\end{frame}

\begin{frame}{}
\end{frame}

\begin{frame}{}
\end{frame}

\begin{frame}{}
\end{frame}

\begin{frame}{}
\end{frame}

\begin{frame}{}
\end{frame}

%Fundamental Concepts
%- How are the experimental measures taken
    %- A few details about the procedure
%- How can we model cell signaling pathways mathematically
    %- kinetics laws
    %- M.M. simplifcations
%- How can we choose between models
    %- Approximate Bayesian Computation
    %- Annealing-Melting Integration
\section{Fundamental Concepts}

%Model Selection
%- Approximate Bayesian Computation
    %- What does it do;
    %- ABC-SMC algorithm
%- A method using Thermodynamic Integration
    %- What does it do;
    %- How to derive the thermodynamic integration
    %- How to estimate this value
\section{Model Selection}


\section{Experiments on Model Selection}

\end{document}


