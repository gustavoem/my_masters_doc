\documentclass[12pt]{article}
\usepackage[portuguese]{babel}
\usepackage[utf8]{inputenc}
\usepackage[usenames,dvipsnames]{color}
\usepackage{setspace}
\usepackage{amsmath}
\usepackage{amsfonts}
\usepackage{amssymb}
\usepackage{mathtools}
\usepackage[top=3cm, bottom=2cm, left=3cm, right=2cm]{geometry}
\usepackage{tikz}
\usepackage{indentfirst}
\usepackage{textcomp}
\usepackage[font={small,it}]{caption}
\title{Relatorio IC}

% packages added by Marcelo
%
\usepackage{lscape}    % for landscape pages
\usepackage{hyperref}  % to allow hyperlinks
\usepackage{booktabs}  % nicer table borders
\usepackage{subfigure} % add subfigures

% My default commands
\newcommand{\foreignword}[1]{\textit{#1}}
\newcommand{\toolname}[1]{\textit{#1}}
%\newcommand{\fieldR}{\mathbb{R}}
\newcommand{\powerset}{\mathcal{P}}
%\newcommand{\probability}{\mathbb{P}}
%\newcommand{\expectation}{\mathbb{E}}
\newcommand{\algname}[1]{\texttt{#1}}
\newcommand{\langname}[1]{\texttt{#1}}
%\newcommand{\varname}[1]{\texttt{#1}}
%\newcommand{\floor}[1]{\lfloor #1 \rfloor}
%\newcommand{\ceil}[1]{\lceil #1 \rceil}
%\newcommand{\mathsc}[1]{{\normalfont\textsc{#1}}}
\newcommand{\forest}{\mathcal{F}}
\newcommand{\pfsnode}[1]{\mathbf #1}
\newcommand{\species}[1]{\textit{#1}}
\newcommand{\gender}[1]{\textit{#1}}

\graphicspath{{./img/}} 

\setstretch{1.5}

\begin{document}

% FAPESP demands the usage of double spacing
%
\doublespacing

\begin{titlepage}
    \vfill 
    \begin{center}
        {\Large Relatório Científico Parcial -- Mestrado\\
         \bigskip
         Processo FAPESP 17/20575-9
        }
        
        \bigskip
        \bigskip
    
        {\LARGE Identificação de vias de sinalização celular baseada em repositórios de cinética de reações bioquímicas}

        \bigskip
        \bigskip
        {\Large {\bf Beneficiário:} \href{mailto:gustavo.estrela.matos@usp.br}{Gustavo Estrela de Matos}\\ 
        
        {\bf Responsável:} \href{mailto:marcelo.reis@butantan.gov.br}{Marcelo da Silva Reis}\\

        \bigskip
        \bigskip
        \bigskip
        \bigskip
        \bigskip
        \bigskip
        \bigskip
Relatório referente aos trabalhos desenvolvidos entre 1 de janeiro e 10 de dezembro de 2018

        \bigskip
        \bigskip
        \bigskip
        \bigskip
        \bigskip
        \bigskip
        \bigskip

Laboratório Especial de Toxinologia Aplicada, Instituto Butantan\\
        \bigskip
        São Paulo, \today\\
        }

        \bigskip
        \bigskip

       

\end{center}
\end{titlepage}


\tableofcontents

\pagebreak



\section{Resumo do Projeto Proposto} \label{sec:resumo} % até 2 páginas

\section{Introdução}
% Uma introdução enxuta do projeto, com outline e objetivos.

\section{Atividades Realizadas}

\subsection{Disciplinas cursadas}
% Mencionar as três matérias feitas
% 1 Tópicos em análise de algoritmos
%   - complementar a análise de algoritmo
% 2 Probabilidade e Inferência Estatística I
% 3 Laboratório de Programação Extrema
% 4 Como ouvinte, aulas de Introdução à Transdução de Sinais 
%   Celulares (Instituto de Química - Universidade de São Paulo)
% 5 Mencionar a matéria do Ronaldo (?)

\subsection{Estudo de literatura em seleção de modelos}
% Achamos melhor começar o projeto resolvendo o desafio de definir uma 
% função de custo apropriada, que leve em consideração a penalização
% por overfitting... Decidimos iniciar pela metodologia menos 
% tradicional, pois imaginamos que a popularidade do AIC implicaria em
% uma utilização mais fácil desse método. Além disso, o conhecimento
% adquirido.

% OBS: o AIC parece ser fácil, apenas definir uma função de 
% verossimilhança e pronto. BIBm é baseado em fator de bayes de 
% verossimilhança marginal, bem mais complexo; precisa de calculo
% de verossimilhança e também de gerar amostra de parametros a 
% posteriori de acordo com os dados observados.


\subsection{Estudos de estimadores de verossimilhança marginal de um 
modelo}
% Falar sobre power-posteriors e sobre como gerar uma amostra de
% theta | D, t.


\subsection{Implementação local do BIBm}
% Em python
% Técnicas já convencionais para desenvolvimento de software como 
% controle de versionamento. Outras técnicas de desenvolvimento ágil 
% foram utilizadas, como TDD, kanban, testes automatizados e 
% integração contínua.

\subsection{Experimentos com o SigNetMS}
% reação enzimática simples
% modelos Girolmi (bioinformatics)
% modelos Kolch (?)


\section{Ativides futuras}
% 1. Finalização de experimentos e testes do SigNetMS
% 2. EQM
% 3. Estruturação do bancos de dados
% 4. População do BD com dados reais das linhagens
%   - Y1
%   - HEK293
%   - HaCat
%   - E6E7-keratinocytes
%   - Erwing Sarcoma
%   - Neuroblastoma
%   - PC12
%   - PC12


\subsection{Participação em conferências}

\pagebreak
\begin{thebibliography}{9} \label{sec:referencias}

\addcontentsline{toc}{section}{Referências}

\bibitem{msreis thesis}
Reis, Marcelo S. ``Minimization of decomposable in U-shaped curves functions defined on poset chains–algorithms and applications." PhD thesis, Institute of Mathematics and Statistics, University of São Paulo, Brazil, (2012).
\end{thebibliography}
\end{document}


