\documentclass[12pt]{article}
\usepackage[portuguese]{babel}
\usepackage[utf8]{inputenc}
\usepackage[usenames,dvipsnames]{color}
\usepackage{setspace}
\usepackage{amsmath}
\usepackage{amsfonts}
\usepackage{amssymb}
\usepackage{mathtools}
\usepackage[top=3cm, bottom=2cm, left=3cm, right=2cm]{geometry}
\usepackage{tikz}
\usepackage{indentfirst}
\usepackage{textcomp}
\usepackage[font={small,it}]{caption}
\title{Relatorio Parcial Mestrado}

% packages added by Marcelo
%
\usepackage{lscape}    % for landscape pages
\usepackage{hyperref}  % to allow hyperlinks
\usepackage{booktabs}  % nicer table borders
\usepackage{subfigure} % add subfigures

% My default commands
\newcommand{\foreignword}[1]{\textit{#1}}
\newcommand{\toolname}[1]{\textit{#1}}
%\newcommand{\fieldR}{\mathbb{R}}
\newcommand{\powerset}{\mathcal{P}}
%\newcommand{\probability}{\mathbb{P}}
%\newcommand{\expectation}{\mathbb{E}}
\newcommand{\algname}[1]{\texttt{#1}}
\newcommand{\langname}[1]{\texttt{#1}}
%\newcommand{\varname}[1]{\texttt{#1}}
%\newcommand{\floor}[1]{\lfloor #1 \rfloor}
%\newcommand{\ceil}[1]{\lceil #1 \rceil}
%\newcommand{\mathsc}[1]{{\normalfont\textsc{#1}}}
\newcommand{\forest}{\mathcal{F}}
\newcommand{\pfsnode}[1]{\mathbf #1}
\newcommand{\species}[1]{\textit{#1}}
\newcommand{\gender}[1]{\textit{#1}}

\graphicspath{{./img/}} 

\setstretch{1.5}

\begin{document}

% FAPESP demands the usage of double spacing
%
\doublespacing

\begin{titlepage}
    \vfill 
    \begin{center}
        {\Large Relatório Científico Parcial -- Mestrado\\
         \bigskip
         Processo FAPESP 17/20575-9
        }
        
        \bigskip
        \bigskip
    
        {\LARGE Identificação de vias de sinalização celular baseada em repositórios de cinética de reações bioquímicas}

        \bigskip
        \bigskip
        {\Large {\bf Beneficiário:} \href{mailto:gustavo.estrela.matos@usp.br}{Gustavo Estrela de Matos}\\ 
        
        {\bf Responsável:} \href{mailto:marcelo.reis@butantan.gov.br}{Marcelo da Silva Reis}\\

        \bigskip
        \bigskip
        \bigskip
        \bigskip
        \bigskip
        \bigskip
        \bigskip
Relatório referente aos trabalhos desenvolvidos entre 1 de janeiro e 10 de dezembro de 2018

        \bigskip
        \bigskip
        \bigskip
        \bigskip
        \bigskip
        \bigskip
        \bigskip

Laboratório Especial de Toxinologia Aplicada, Instituto Butantan\\
        \bigskip
        São Paulo, \today\\
        }

        \bigskip
        \bigskip

       

\end{center}
\end{titlepage}


\tableofcontents

\pagebreak



\section{Resumo do Projeto Proposto} \label{sec:resumo} % até 2 páginas
A construção de modelos funcionais é uma técnica comum para se 
estudar vias de sinalização celular e, quando a via estudada é pouco
conhecida, é possível que os modelos já propostos sejam incompletos, 
tornando necessário a sua modificação.
Lulu Wu apresentou em 2015, em sua dissertação de mestrado, um método 
para  modificar sistematicamente modelos funcionais, adicionando a estes
interações extraídas de repositórios como KEGG. Entretanto, esta 
metodologia apresentou limitações: a primeira é a incompletude do banco 
de dados de interações criado, que extraia informações apenas do 
repositório KEGG; a segunda, a falta de informações sobre constantes 
de velocidade de interações, que podem ser extraídas de repositórios 
como BioNumbers; a terceira, a dinâmica do algoritmo de busca, 
incremental, que pode não achar o mínimo global; e a última, a 
penalização na complexidade dos modelos, que era feita de maneira 
aleatória. Propomos neste trabalho enfrentar as limitações encontradas
pela metodologia de Lulu, criando um banco de dados de interações mais
completo e também novas funções de custo que sejam capazes de 
penalizar modelos mais complexos (como critério de informação Akaike e 
{\em Bayesian inference-based modeling}); esta penalização deve induzir,
em cadeias do  espaço de busca, curvas em u no custo dos modelos, 
portanto também propomos a criação de novos algoritmos de busca que 
explorem essa característica da função de custo. Por fim, esperamos 
testar nossa metodologia na identificação de vias de sinalização celular 
da linhagem tumoral murina Y1.

\section{Introdução}
% Uma introdução enxuta do projeto, com outline e objetivos.
Vias de sinalização celular podem ser simuladas por modelos dinâmicos
computacionais e, mais especificamente, modelos que descrevem a 
concentração de espécies químicas ao longo do tempo são chamados de 
modelos funcionais. Neste projeto, trabalhamos com modelos funcionais
que descrevem as mudanças de concentrações de espécies químicas através
de equações diferenciais ordinárias (EDOs). Estes modelos, quando não 
sofrem de sobreajuste (\foreignword{overfitting}), se tornam 
interessantes quando são capazes de reproduzir dados observados em 
experimentos biológicos, pois dão a estes modelos a qualidade preditiva.

% Agora falamos do problema de identificação de via de sinalização
% celular.
% Quebramos o problema em duas partes:
% 1. Determinar a topologia da via de sinalização
% 2. Determinar constantes de velocidades de interações
% Propomos solucionar os dois problemas retirando dados de bancos de 
% dados de biologia. Mais especificamente, em uma de nossas metodologias 
% propostas tratamos constantes de velocidade como parâmetros aleatórios 
% do sistema, o que dignifica nossa abordagem como Bayesiana.

% Depois falamos que por muitas vezes um mapa estático pode não ter uma
% topologia compatível com o experimento biológico em questão (pode ser
% que falte ou sobre interações). Por isso, torna-se necessário 
% modificar de maneira sistemática. Propomos então neste projeto 
% transformar o problema de identificação de vias de sinalização celular
% em um problema de otimização.

O problema de criar modelos funcionais capazes de explicar resultados
de experimentos biológicos com o mínimo de sobreajuste é chamado de 
{\em problema de identificação de vias de sinalização celular}. Podemos 
separar este problema em duas etapas principais. A primeira diz respeito
a escolha da topologia da via de sinalização, o que é equivalente a 
escolher quais interações químicas são relevantes para o experimento em
questão. A segunda etapa consiste em escolher valores para os parâmetros
do sistema de EDOs do modelo funcional; estes valores são constantes de 
velocidade de interações e/ou concentrações iniciais de espécies 
químicas.

Em casos em que o experimento biológico ou a via de sinalização são 
muito estudadas, é possível que ambas as etapas descritas anteriormente
possam ser resolvidas com pesquisas na literatura. Em casos que isto não
é possível, torna-se uma solução recorrer a bancos de dados de biologia.
Para a primeira etapa, podemos consultar bancos como o 
\href{http://www.genome.jp/kegg/}{Kyoto Encyclopedia of Genes and Genomes (KEGG)}
~\cite{Kanehisa2000kegg}, que contém mapas estáticos; e para a segunda,
podemos consultar bancos como o 
\href{https://www.ebi.ac.uk/biomodels-main/}{BioModels}~\cite{le2006biomodels}
.

Entretanto, os mapas estáticos disponíveis nestes bancos podem ainda ser
incompletos ou muito grandes para o experimento biológico em questão. 
Desta maneira, torna-se importante criar uma maneira sistemática de se
modificar modelos funcionais a fim de faze-los explicar o experimento 
biológico sem sobreajuste. Propomos fazer estas modificações tratando
o problema de identificação de vias de sinalização como um problema de
otimização combinatória, considerando como espaço de busca possíveis 
topologias para o modelo funcional e usando como função de custo alguma
métrica que represente a qualidade deste modelo ao reproduzir o 
experimento biológico de interesse.

Considere $S$ um conjunto de interações químicas. O conjunto de todas
as possíveis escolhas de interações relevantes em $S$ corresponde ao
conjunto potência de $S$, $\powerset(S)$, que é o espaço de busca do
problema de otimização que estamos interessados. Se chamamos a nossa 
métrica de qualidade de modelo funcional de $c$, transformamos nosso 
problema em uma instância do problema de seleção de características. 
Assim, após definida a função de custo $c$ e devidamente coletado e 
armazenado o conjunto $S$, propomos resolver instâncias do problema de 
identificação de vias de sinalização celular no arcabouço 
{\em featsel} ~\cite{Reis2017featsel}. 

%TODO: adicionar imagem do Kolch de escolha de modelo e ao lado um 
% reticulado booleano equivalente.

\section{Atividades Realizadas}

\subsection{Disciplinas cursadas}
% Mencionar as três matérias feitas
% 1 Tópicos em análise de algoritmos
%   - complementar a análise de algoritmo
% 2 Probabilidade e Inferência Estatística I
% 3 Laboratório de Programação Extrema
% 4 Como ouvinte, aulas de Introdução à Transdução de Sinais 
%   Celulares (Instituto de Química - Universidade de São Paulo)
% 5 Mencionar a matéria do Ronaldo (?)

Durante o período de março a dezembro de 2018, foram cursadas pelo
beneficiários três disciplinas. No primeiro semestre, 
\begin{itemize}
    \item{Tópicos em Análise de Algoritmos:} nesta disciplinas são
        abordadas técnicas para análise de algoritmos e para solução
        de problemas. Muitos problemas abordados nesta disciplina são
        de otimização combinatória, assim como o problema de seleção de 
        características, que propomos utilizar neste projeto.
    \item{Probabilidade e Inferência Estatística I:} esta disciplina
        faz parte do departamento de Estatística do Instituto de 
        Matemática e Estatística (IME-USP). A disciplina é dividida em
        dois módulos: o primeiro módulo aborda probabilidade, enquanto
        o segundo aborda inferência estatística, tanto do ponto de vista
        clássico quanto do ponto de vista Bayesiano. Esta disciplina
        foi necessária para que o beneficiário pudesse entender as 
        duas funções de custo propostas para este trabalho: {\em 
        Akaike's Information Criterion}, uma abordagem clássica; e 
        {\em Bayesian inference-based modeling} (BIBm)~\cite{Xu2010}, 
        uma abordagem Bayesiana.
\end{itemize}

No segundo semestre, apenas a disciplina Laboratório de Programação 
Extrema foi cursada. Nesta disciplina, projetos com clientes reais são
desenvolvidos pelos alunos usando a metodologia de programação extrema, 
uma metodologia ágil para desenvolvimento de sistemas. Algumas das 
técnicas ensinadas na disciplina estão sendo usadas no desenvolvimento
deste projeto, como controle de versões, desenvolvimento orientado a 
testes e integração contínua. 

Além disso, no segundo semestre, o beneficiário frequentou como ouvinte
a disciplina Introdução à Transdução de Sinais Celulares, no Instituto 
de Química da Universidade de São Paulo. Frequentando esta disciplina,
o beneficiário pode se familiarizar com os processos bioquímicos 
envolvidos em uma via de sinalização celular.

\subsection{Estudo de literatura em seleção de modelos}
% Achamos melhor começar o projeto resolvendo o desafio de definir uma 
% função de custo apropriada, que leve em consideração a penalização
% por overfitting... Decidimos iniciar pela metodologia menos 
% tradicional, pois imaginamos que a popularidade do AIC implicaria em
% uma utilização mais fácil desse método. Além disso, o conhecimento
% adquirido.

% OBS: o AIC parece ser fácil, apenas definir uma função de 
% verossimilhança e pronto. BIBm é baseado em fator de bayes de 
% verossimilhança marginal, bem mais complexo; precisa de calculo
% de verossimilhança e também de gerar amostra de parametros a 
% posteriori de acordo com os dados observados.

Apesar de propormos iniciar o projeto pela criação do banco de dados
necessário para armazenar interações químicas relevantes assim como 
constantes de velocidades, decidimos iniciar o projeto pelo primeiro
desafio científico reconhecido na proposta do projeto: definir uma 
função de custo apropriada para a seleção de modelos. Acreditamos que 
esta mudança tenha sido benéfica pois a criação do banco de dados 
necessitaria de um maior entendimento, pelo beneficiário, dos processos
bioquímicos modelados. Assim, o beneficiário foi capaz de progredir
na implementação de uma função de custo ao mesmo tempo em que se 
familiarizava com os experimentos biológicos que seriam modelados. 
Essa familiarização se deu pela participação do beneficiário em 
seminários promovidos pelo laboratório Especial de Toxinologia Aplicada,
no Instituto Butantan, assim como sua participação como ouvinte na 
disciplina Introdução à Transdução de Sinais Celulares, no Instituto de
Química da USP.

Assim, iniciamos o desenvolvimento deste projeto estudando funções de
custo para modelos funcionais. Mais especificamente, decidimos começar
estudando o {\em Bayesian inference-based modeling} (BIBm)~\cite{Xu2010}.
De maneira superficial, nesta metodologia a qualidade de um modelo 
funcional é medida pela estimativa da probabilidade dos dados do 
experimento serem observados dado que o modelo em questão gera os dados 
observados; isto é, assumindo que o modelo avaliado representa bem o
experimento biológico, medimos a probabilidade das observações feitas
no experimento. Mais formalmente, dado um conjunto de observações
experimentais $D$ e um modelo $M$, a métrica aplicada no BIBm é o valor
de $p (D | M)$.

% O theta é uma variável aleatória e o valor p (D | M) 
Por ser uma abordagem Bayesiana, esta metodologia considera que o vetor
de parâmetros de um modelo $M$ é um vetor aleatório $\theta_M$, de um
espaço paramétrico $\Theta_M$. Portanto, podemos escrever
\begin{equation}
    p (D | M) = \int_{\Theta_M} p (D | M, \theta_M) d\theta_M
\end{equation}
Ou seja, a métrica $p (D|M)$ é obtida ao marginalizar a verossimilhança
$p (D | M, \theta_M)$. Entretanto, como estamos trabalhando com sistemas
de EDOs, a verossimilhança muitas vezes não pode ser determinada 
analiticamente, assim como a integral que resulta na verossimilhança
marginal $p (D | M)$. Para enfrentar este problema, a metodologia 
aplicada pelo BIBm usa como estratégia funções intermediárias entre a 
priori, $p (\theta_M)$, e a posteriori, $p (D | \theta_M)$, para obter 
um estimador de $P (D|M)$.

\subsection{Estudos de estimadores de verossimilhança marginal de um 
modelo}
% Falar sobre power-posteriors e sobre como gerar uma amostra de
% theta | D, t.



\subsection{Implementação local do BIBm}
% Em python
% Técnicas já convencionais para desenvolvimento de software como 
% controle de versionamento. Outras técnicas de desenvolvimento ágil 
% foram utilizadas, como TDD, kanban, testes automatizados e 
% integração contínua.

\subsection{Experimentos com o SigNetMS}
% reação enzimática simples
% modelos Girolmi (bioinformatics)
% modelos Kolch (?)


\section{Ativides futuras}
% 1. Finalização de experimentos e testes do SigNetMS
% 2. EQM
% 3. Estruturação do bancos de dados
% 4. População do BD com dados reais das linhagens
%   - Y1
%   - HEK293
%   - HaCat
%   - E6E7-keratinocytes
%   - Erwing Sarcoma
%   - Neuroblastoma
%   - PC12
%   - ToyModels
% 5. Wrapper do SigNetMS ao featsel
% 6. Testes com toy models
% 7. Aplicação da metodologia em linhagens celulares
% 8. Análise  dos resultados
% 9. Elaboração da dissertação
% 10. Defesa

\subsection{Outras atividades acadêmicas}
% Santiago (outorgado pela FAPESP)
% Apresentar resultados em conferência internacional (ICSB)
% Apresentar resultados em outra conferencia nacional (X-meeting e RCA)

\subsection{Participação em conferências}


\addcontentsline{toc}{section}{Referências}
\bibliographystyle{unsrt} 
\bibliography{bib-relatorio}
\end{document}


