%begin-include
As we presented of the last sections of this text, we are close to
defining a ranking framework for models of signaling pathway. To achieve
our goal, of creating a methodology for identification of signaling 
pathway using an approach based on the feature selection problem, a few
activities need to be accomplished. These activities include the
definition of a reliable method of model ranking, the construction of
a relational database with chemical reactions information, the creation
of search algorithms in the space of models, and finally the validation
of the proposed methodology.

The first activity that should be tackled is the definition of a 
framework for ranking models of signaling pathways. This framework will
use either the ABC-SysBio or SigNetMS software. The first has been 
tested already and the last is still being tested and analyzed. 
Therefore, to define the framework we need to finish the tests of 
SigNetMS and decide which software is more adequate for our use. After
that, it is of our interest to propose changes in the implementation of 
the chosen software in order to achieve better computational times, 
using, as an example, distributed computation or computation in multiple 
processors, including graphics processing units (GPUs).

After defining the model ranking framework, we will define the 
relational database of chemical reactions, which will be used to define 
the search space of the feature selection problem. This database should
be able to store interactions between chemical species as well as 
reaction rate constants.  This database will be populated with 
information from other databases available, such as 
SABIO-RK~\cite{Wittig2011} and BRENDA~\cite{Schomburg2004}. Interactions 
of the database will be used to propose new model hypothesis for the 
experimental data, while the reaction rate constants will be used to 
define the prior distribution of model parameters (distributions should 
have high mass concentration around plausible values for reaction rate 
constants).

With a defined model ranking framework and a database with informations
of chemical reactions we will then work on the definition of the
search space and cost function that will allow us to solve the 
identification of signaling pathways as a feature selection problem.
The cost function should be implemented on featsel, a framework that 
will also allow us to implement and different search algorithms of 
feature selection that we create to solve our problem. The featsel
framework can also benchmark search algorithms, and this feature is 
going to be used by us to improve algorithms and choose the ones with 
better performance.

Using the cost function and search space that we defined on the featsel
framework, we will then be able to test our methodology on well known
signaling pathways that were previously modeled. One possible approach 
to test the methodology is to use an incomplete model of a signaling 
pathway and a set of experimental data to test if the created 
methodology can extend the incomplete model to a model that can 
successfully reproduce the data. A similar test is to give an overly 
complex (or even a model with spurious interactions) model of a 
signaling pathway and a set of experimental data to test if the 
methodology can prune the unnecessary (or spurious) interactions. 

Finally, after testing and validating, we can apply our methodology on
problems that are relevant for other researchers of the our laboratory, 
mostly problems related to ERK signaling pathways of tumor cell lines
Y1 and HEK293.

From now to the end of the project we will also produce a dissertation
with our work. We also intend to produce a manuscript that can be sent 
for publication in a journal of Systems Biology. Moreover, the student
still need to take one class on the second semester of 2019 to fulfill
the credits required by the Computer Science MSc program. 


\section{Activities description}
\label{sec:future_activities}
\begin{itemize}
    \item{\bf Activity 1:} Finalize experiments with SigNetMS.
    \item{\bf Activity 2:} Determine, between ABC-SysBio and SigNetMS,
        which is best to rank models.
    \item{\bf Activity 3:} Studies of databases of chemical kinetics 
        such as SABIO-RK~\cite{Wittig2011} and 
        BRENDA~\cite{Schomburg2004}.
    \item{\bf Activity 4:} Creation of a relational database of chemical 
        interactions that is able to store the topology and rate 
        constants of reactions gathered from chemical kinetics 
        databases.
    \item{\bf Activity 5:} Implementation of a feature selection cost
        function and definition of the search space on featsel, which 
        will allows us to solve the identification of signaling pathways 
        as a feature selection problem.
    \item{\bf Activity 6:} Creation of new algorithms of feature 
        selection for the problem of identification of signaling 
        pathways.
    \item{\bf Activity 7:} Testing of the methodology developed on known
        pathways.
    \item{\bf Activity 8:} Application of the method in ERK signaling 
        pathways of tumor cell lines Y1 and HEK293.
\end{itemize}

\section{Plan of Work for Future Activities}
On Table~\ref{tab:Gantt} we show Gantt diagram with our estimate of 
start and finish of each future activity of the project.

\begin{table}[H]
\centering
  \begin{ganttchart}[x unit=0.38mm, y unit chart=7mm, time slot format=isodate, bar/.append style={fill=gray!30},vgrid={
      draw=none,
      draw=none,
      dotted
    },]{2019-5-13}{2019-12-31}
  \gantttitlecalendar{year}\\
  \gantttitle{May}{20}
  \gantttitle{Jun}{30}
  \gantttitle{Jul}{31}
  \gantttitle{Aug}{30}
  \gantttitle{Sep}{31}
  \gantttitle{Oct}{30}
  \gantttitle{Nov}{31}
  \gantttitle{Dec}{30} \\
  \ganttbar{Activity 1}{2019-5-13}{2019-5-30}\\
  \ganttbar{Activity 2}{2019-5-24}{2019-6-15}\\
  \ganttbar{Activity 3}{2019-6-1}{2019-6-15}\\
  \ganttbar{Activity 4}{2019-6-15}{2019-7-15}\\
  \ganttbar{Activity 5}{2019-7-15}{2019-8-15}\\
  \ganttbar{Activity 6}{2019-8-15}{2019-10-30}\\
  \ganttbar{Activity 7}{2019-9-20}{2019-10-30}\\
  \ganttbar{Activity 8}{2019-11-1}{2019-12-30}
  \end{ganttchart}
\caption{Gantt diagram containing a proposed schedule for all activity
    described on Section~\ref{sec:future_activities}.} 
\label{tab:Gantt}
\end{table}
